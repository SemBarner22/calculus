% \chapter{Основные понятия}

\section{Некоторые базовые понятия}

\begin{definition}
    \textit{Отображением} из множества $X$ в множество $Y$ называется
    отношение $F \subseteq X \times Y$, для которого
\[
    \forall x \in X~\exists! y \in Y \colon ~(x, y) \in F
\]
Обозначается $F \colon X \to Y$ или $X \xrightarrow{F} Y$. Сам факт того, что $(x, y) \in F$ обозначается $f(x) = y$. $X$ называют \textit{областью определения}, а $Y$ --- \textit{областью значений} $f$.
\end{definition}

\begin{definition}
    Отображение $f \colon X \lra Y$ называется \textit{инъективным}, если для него выполняется
\[
    x_1 \neq x_2 \Lra f(x_1) \neq f(x_2)
\]
Иначе говорят, что $f$ --- \textit{1-1 отображение}, и часто обозначают
$f \colon X \xrightarrow{1-1} Y$.
\end{definition}

\begin{definition}
    Отображение $f \colon X \to Y$ называется \textit{сюръективным}, если для него выполняется
\[
    \forall y \in Y~ \exists x \in X \colon~ f(x) = y
\]
Такие отображения называют отображениями \textit{на}, и часто обозначают
$f \colon X \xrightarrow{\text{на}} Y$.

\end{definition}

\begin{definition}
    Отображение $f \colon X \to Y$ называют \textit{биекцией}, если оно одновременно сюръективно и инъективно, иначе говоря
\[
    \forall y \in Y~ \exists! x \in X \colon~ f(x) = y
\]
Такие отображения, по аналогии с предыдущими определениями,
называют \textit{1-1 на отображениями}, и часто обозначают
$f \colon X \xrightarrow[\text{на}]{1-1} Y$.
\end{definition}

\begin{definition}
    \textit{Образом} множества $A \subseteq X$ при отображении $f \colon X \to Y$
    называют множество
\[
    f(A) \defeq \{\, f(x) \mid x \in A \,\}
\]
\end{definition}

\begin{definition}
    \textit{Прообразом} множества $B \subseteq Y$ при отображении
    $f \colon X \to Y$ называют множество
\[
    f^{-1}(B) \defeq \{\, x \in X \mid f(x) \in B \,\}
\]
\end{definition}

\begin{definition}
    \textit{Обратимым} называется отображение $f \colon X \to Y$, для которого
    существует обратное относительно композиции отображение
    $f^{-1} \colon Y \to X$, для которого выполняется
\[
    f^{-1} \circ f = id_X
\]
\end{definition}

\begin{theorem}[Свойства прообраза]
    Пусть $f \colon X \to Y$, \,$A, B \subseteq X$. Тогда справедливо
    \begin{enumerate}
        \item $f^{-1}(A \cup B) = f^{-1}(A) \cup f^{-1}(B)$
        \item $f^{-1}(A \cap B) = f^{-1}(A) \cap f^{-1}(B)$
    \end{enumerate}
\end{theorem}
\begin{proof}
    Без доказательства (очевидно).
\end{proof}

\begin{theorem}
    $f$ биективно $\Llra$ $f$ обратимо
\end{theorem}
\begin{proof}
    Без доказательства (тривиально).
\end{proof}

\newpage

\begin{definition}
    \textit{Полем} называется тройка $\langle X, + \colon X \times X \to X,
    \cdot \colon X \times X \to X \rangle$, \, где $X$ --- множество,
    удовлетворяющая аксиомам поля:
    \begin{itemize}
        \item[+G1] $\a + (\b + \g) = (\a + \b) + \g$
        \item[+G2] $\exists 0 \in X \colon~ \a + 0 = 0 + \a = \a$
        \item[+G3] $\exists -\a \colon~ \a + -\a = 0$
        \item[+G4] $\a + \b = \b + \a$
        \item[$\cdot$G1] $\a(\b\g) = (\a\b)\g$
        \item[$\cdot$G2] $\exists 1 \in X \colon~ 1\a = \a1 = \a$
        \item[$\cdot$G3] $\a \neq 0 \Rightarrow \exists \a^{-1} \colon~
         \a\a^{-1} = 1$
        \item[$\cdot$G4] $\a\b = \b\a$
        \item[D] $(\a + \b)\g = \a\g + \b\g$
    \end{itemize}
Для любых $\a, \b, \g \in X$. Аксиомы $+$G1-4 задают на $X$ структуры
абелевой группы по $+$, аксиомы $\cdot$G1-4 задают на $X \setminus \{0\}$
структуру абелевой группы по $\cdot$, аксиома дистрибутивности D связывает $+$
и $\cdot$.
\end{definition}
\begin{examples}
    \enewline
    \begin{enumerate}
        \item $\mathbb{R}$ --- поле. В дальнейшем можно под произвольным полем
        понимать $\mathbb{R}$, общность от этого сильно не пострадает.
        \item $\mathbb{Z}_p$ тогда и только тогда поле, когда $p$ --- простое.
    \end{enumerate}
\end{examples}

\begin{definition}
    \textit{Векторным (линейным) пространством над полем} $K$ называют тройку
    $\langle V, +_V \colon V \times V \to V, \cdot_V \colon V \times K \to V
    \rangle$, где $V$ --- множество, удовлетворяющюю аксиомам:
\begin{enumerate}
    \item[$+_V$G1] $\x + (\y + \z) = (\x + \y) + \z$
    \item[$+_V$G2] $\exists \elemvec{0} \in X \colon~ \x + \elemvec{0} =
    \elemvec{0} + \x = \x$
    \item[$+_V$G3] $\exists -\x \colon~ \x + -\x = \elemvec{0}$
    \item[$+_V$G4] $\x + \y = \y + \z$
    \item[V1] $(\a + \b)\x = \a\x + \b\x$
    \item[V2] $(\a\b)\x = \a(\b\x)$
    \item[V3] $\a(\x + \y) = \a\x + \a\y$
    \item[V4] $1_K \x = \x$
\end{enumerate}
Для любых $\a, \b \in K$, $\x, \y \in V$.
\end{definition}
\begin{example}
    $K^n \defeq \underbrace{K \oplus K \oplus \ldots \oplus K}_{n}$
        --- векторное пространство, которому изоморфны все векторные
        пространства над полем $K$ размерности $n$. Мы ограничимся
        рассмотрением $\mathbb{R}^n$.
\end{example}

\begin{theorem}(Закон Де-Моргана)
    Пусть $\{X_\a\}_{\a \in A}$ --- семейство множеств, и $Y$ --- множество.
    Тогда справедливо
\[
    Y \setminus \Big( \bigcup_{\a \in A}{X_\a} \Big) =
    \bigcap_{\a \in A}{\big( Y \setminus X_\a \big)}
\]
\end{theorem}
\begin{proof}
    Докажем, что $z$ принадлежит левой части $\Llra z$ принадлежит правой
    части:
\begin{align*}
    z \in Y \setminus \Big( \bigcup_{\a \in A}{X_\a} \Big) \Llra
    z \in Y \wedge z \notin \bigcup_{\a \in A}{X_\a} \Llra
    z \in Y \wedge \forall \a \in A~z \notin X_\a \Llra \\
    \forall \a \in A~ z \in Y \wedge z \notin X_\a \Llra
    z \in \bigcap_{\a \in A}{\big( Y \setminus X_\a \big)}
\end{align*}
\end{proof}

\begin{theorem}(Неравенство Бернулли)
    $(1 + x)^n \geqslant 1 + nx$ при $x \geqslant -1$, $n \in \mathbb{N}$
\end{theorem}
\begin{proof}
    \enewline
    \begin{itemize}
        \item[i)] База индукции при $n = 1$: $1 + x \geqslant 1 + x$
        \item[ii)] Индукционный переход:
        \begin{align*}
            (1 + x)^n = (1 + x)(1 + x)^{n-1} \geqslant (1 + x)(1 + (n - 1)x) = \\
            1 + x + (n - 1)x + (n - 1)x^2 = 1 + nx + (n - 1)x^2 \geqslant 1 + nx
        \end{align*}
    \end{itemize}
\end{proof}

\begin{theorem}(Неравенство КБШ)
\[
    \Bigg|\sum_{i=1}^{n}{a_i b_i}\Bigg| \leqslant \sqrt{\sum_{i=1}^{n}{a_i^2}}\sqrt{\sum_{i=1}^{n}{b_i^2}}
\]
\end{theorem}
\begin{proof}
    \begin{align*}
        0 \leqslant \frac{1}{2}\sum_{i,k=1}^{n}{(a_i b_k - a_k b_i)^2} =
        \frac{1}{2}\sum_{i,k=1}^{n}{(a_i^2 b_k^2 + a_k^2 b_i^2 - 2a_i b_i a_k
        b_k)} = \\ \frac{1}{2}\Bigg[ \Big(\sum_{i=1}^{n}{a_i^2}
        \sum_{k=1}^{n}{b_k^2} \Big) + \Big(\sum_{i=1}^{n}{b_i^2}
        \sum_{k=1}^{n}{a_k^2} \Big) - 2\Big(\sum_{i=1}^{n}{a_i b_i} \Big)\Big(
        \sum_{k=1}^{n}{a_k b_k}\Big) \Bigg] = \\
        \sum_{i=1}^{n}{a_i^2}\sum_{i=1}^{n}{b_i^2} -
        \Big(\sum_{i=1}^{n}{a_i b_i} \Big)^2
    \end{align*}
\end{proof}

\newpage
