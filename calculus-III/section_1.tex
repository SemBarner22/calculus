% preamble change fix
\newcommand{\x}{\ex}
\newcommand{\y}{\ey}
\newcommand{\h}{\eh}
\newcommand{\w}{\ew}
\newcommand{\z}{\ez}
\newcommand{\Od}{\cO}
\newcommand{\N}{\bN}
\renewcommand{\C}{\bC}
% /preamble change fix

\chapter{Многомерный анализ}

\section{Сведения из линейной алгебры}

\begin{definition}
    $L(\Rm, \Rn)$ --- \textit{пространство линейных отображений} из $\Rm$ в $\Rn$
\end{definition}

\begin{definition}
    Элементы $L(\Rm, \Rn)$ называются \textit{операторами}
\end{definition}

\begin{definition}
    \textit{Нормой} на множестве $X$ называется отображение
     $\norm{\phantom{x}} \colon X \to \mathbb{R}$, удовлетворяющее свойствам
     \begin{itemize}
         \item[i)] $\norm{\x} \geqslant 0$, $\norm{\x} = 0 \Llra \x = 0$
         \item[ii)] $\norm{\a\x} = |\a|\norm{\x}$
         \item[iii)] $\norm{\x + \y} \leqslant \norm{\x} + \norm{\y}$
     \end{itemize}
\end{definition}

\begin{definition}
    \textit{Нормированным пространством} называется пара $\langle X,
    \norm{\phantom{x}} \rangle$
\end{definition}

\begin{remark}
    Отображение, задаваемое формулой $d(\x, \y) = \norm{\x - \y}$ является
    метрикой. Поэтому все нормированные пространства сразу можно считать
    и метрическими.
\end{remark}

\begin{theorem}(Об эквивалентности норм в конечномерных пространствах)

    Пусть $V$ --- конечномерное линейное пространство, а $\norm{\phantom{x}}_1$
    и $\norm{\phantom{x}}_2$ --- нормы на $V$. Тогда
\[
    \exists c, C > 0\colon~ c\norm{\x}_1 \leqslant \norm{\x}_2 \leqslant
    C\norm{\x}_1
\]
\end{theorem}
\begin{proof}
    Пусть $\displaystyle \norm{x} = \norm{c_1 e_1 + c_2 e_2 + \ldots + c_n e_n}
    \defeq \sqrt{\sum_{i=1}^{n}{c_i^2}}$.

    \begin{itemize}
        \item[i)] $\norm{\phantom{x}}$ --- норма.
            \begin{itemize}
                \item[$\cdot$] $\displaystyle \sqrt{\sum_{i=1}^{n}{c_i^2}}
                \geqslant 0$ --- очевидно
                \item[$\cdot$] $\displaystyle \sqrt{\sum_{i=1}^{n}{(\a c_i)^2}}
                = \a\sqrt{\sum_{i=1}^{n}{c_i^2}}$ --- очевидно
                \item[$\cdot$] $\displaystyle \sqrt{\sum_{i=1}^{n}{(c_i +
                b_i)^2}} \leqslant \sqrt{\sum_{i=1}^{n}{c_i^2}} +
                \sqrt{\sum_{i=1}^{n}{b_i^2}}$ --- неравенство Минковского
            \end{itemize}
        \item[ii)] Проверим теперь, что все нормы на $V$ эквивалентны
        $\norm{\phantom{x}}$.
\[
        \norm{x_1 e_1 + \ldots + x_n e_n}_1 \leqslant \sum_{i=1}^{n}{\norm{x_i
        e_i}_1} = \sum_{i=1}^{n}{|x_i| \norm{e_i}_1} \leqslant_{\text{КБШ}}
        \sqrt{\sum_{i=1}^{n}{|x_i|^2}} \sqrt{\sum_{i=1}^{n}{\norm{e_i}_1^2}} =
        c\sqrt{\sum_{i=1}^{n}{|x_i|^2}}
\]
        тогда
\[
        |\norm{\x}_1 - \norm{\y}_1| \leqslant \norm{\x - \y}_1 \leqslant
        c \norm{\x - \y}
\]
        Поэтому $\norm{\phantom{x}}_1$ --- непрерывное отображение $\Rm \to \R$.
        Раз так, найдем максимум и минимум этого отображения на сфере (компакт,
        поэтому максимум и минимум реализуются). Пусть
        \begin{gather*}
            c_1 := \min_{\x \in S^n}{\norm{\x}_1} \\
            c_2 := \max_{\x \in S^n}{\norm{\x}_1}
        \end{gather*}
        Ни $c_1$, ни $c_2$ не равны нулю (потому что норма равна нулю только на
        нулевом векторе, который сфере не принадлежит). Тогда
        \begin{gather*}
            \norm{\x}_1 = \norm{\frac{\x}{\norm{\x}}}_1 \norm{\x} \geqslant
            c_1 \norm{\x} \\
            \norm{\x}_1 = \norm{\frac{\x}{\norm{\x}}}_1 \norm{\x} \leqslant
            c_2 \norm{\x}
        \end{gather*}
		\item Покажем теперь, что из этого следует утверждение теоремы.
			Пусть
\begin{align*}
	c^1_1\norm{x} \leqslant \norm{x}_1 \leqslant c^1_2\norm{x} \\ 
	c^2_1\norm{x} \leqslant \norm{x}_2 \leqslant c^2_2\norm{x}
\end{align*}
			Тогда
\[
	\frac{c^2_1}{c^1_2} \norm{x}_1 \leqslant c^2_1\norm{x} \leqslant \norm{x}_2 
	\leqslant c^2_2\norm{x} \leqslant \frac{c^2_2}{c^1_1}\norm{x}_1	
\]
    \end{itemize}
\end{proof}

\begin{definition}
    \textit{Нормой оператора} называется отображение $\norm{\phantom{x}} \colon L(\Rm, \Rn) \to \R$
\[
    \norm{\mathcal{A}} \defeq \sup_{\x \in S^m}{\norm{\mathcal{A}x}_{\Rn}}
\]
\end{definition}

\begin{remark}
    $\displaystyle \sup_{\norm{x} = 1}{\cA \x} = \sup_{\norm{x} \leqslant 1}{\cA
    \x}$
\end{remark}

\begin{theorem}(Пространство линейных операторов)

    $\norm{\phantom{x}} \colon L(\Rm, \Rn) \to \R$ --- действительно норма.
\end{theorem}
\begin{proof}
\[
    \norm{\mathcal{A}(x_1 e_1 + \ldots + x_n e_n)} \leqslant
    \sum_{i=1}^{n}{|\x| \norm{\mathcal{A}e_i}} \leqslant_{\text{КБШ}}
    \norm{\x} \sum_{i=1}^{n}{\norm{\mathcal{A}e_i}}
\]
    Поэтому супремум конечен для всех элементов $L(\Rm, \Rn)$, то есть
    отображение определено корректно. Проверим свойства нормы:

    \begin{itemize}
        \item[i)] $\norm{\cA} = 0 \Llra \forall \x \in S^n~\cA \x = 0
        \Llra \cA = 0$. Неотрицательность очевидна.
        \item[ii)] $\displaystyle \norm{\a \cA} = \sup_{\x \in S^n}{\norm{\a \cA
        \x}} = \sup_{\x \in S^n}{|\a| \norm{\cA \x}} = \a \norm{\cA}$
        \item[iii)] $\displaystyle \norm{\cA + \cB} = \sup_{\x \in S^n}{\cA \x +
        \cB \x} \leqslant \sup_{\x \in S^n}{\cA \x} + \sup_{\x \in S^n}{\cB \x}$
    \end{itemize}
\end{proof}

\begin{theorem}(Липшицевость линейных опрераторов)

    $\cA \in L(\Rm, \Rn) \Lra \cA$ --- липшицево
\end{theorem}
\begin{proof}
\[
    \norm{\cA \x - \cA \y} = \norm{\cA(\x - \y)} = \norm{\cA \left(\frac{\x -
    \y}{\norm{\x - \y}}\right)} \norm{\x - \y} \leqslant \norm{\cA} \norm{\x - \y}
\]
\end{proof}

\begin{theorem}(О произведении линейных операторов)

    $\cA \in L(\Rm, \Rn)$, $\cB \in L(\Rn, \mathbb{R}^l)$, тогда
    $\cB \cA \in L(\Rm, \mathbb{R}^l)$, причем $\norm{\cB \cA} \leqslant
    \norm{\cB} \norm{\cA}$
\end{theorem}
\begin{proof}
    \begin{align*}
        \norm{\cB \cA} &= \sup_{\x \in S^n}{\cB(\cA \x)} \\
        &= \sup_{\x \in S^n}{\left(\norm{\cA \x} \cdot \cB\left(\frac{\cA
        \x}{\norm{\cA \x}}\right)\right)} \\
        &\leqslant \sup_{\x \in S^n}{\norm{\cA \x}} \cdot
        \sup_{\x \in S^n}{\cB\left(\frac{\cA \x}{\norm{\cA \x}}\right)} \\
        &\leqslant \norm{A} \norm{B}
    \end{align*}
\end{proof}

\begin{definition}
    $\O_m$ --- \text{пространство обратимых линейных операторов} на $\Rm$
\end{definition}

\begin{lemma}(Критерий обратимости линейного оператора)

    $\cA \in L(\Rm, \Rn)$ обратим тогда и только тогда, когда $m = n$ и
    $\Ker(\cA) = 0$
\end{lemma}
\begin{proof}
    Линейная алгебра.
\end{proof}

\begin{lemma}(Об условиях, эквивалентных обратимости оператора)

    $\cA \in L(\Rm, \Rm)$ обратим $\Llra \exists c>0~ \forall \x~ \norm{\cA \x}
    \geqslant c\norm{\x}$, причем $\displaystyle \norm{\cA^{-1}} \leqslant
    \frac{1}{c}$
\end{lemma}
\begin{proof}
\[
    \norm{\cA^{-1} \y} \leqslant \norm{\cA^{-1}} \norm{\y} \Lra \norm{\y} =
    \norm{\cA \x} \geqslant \frac{1}{\norm{\cA^{-1}}} \norm{\x}
\]
\end{proof}

\begin{theorem}(Об обратимости оператора, близкого к обратимому)

    $\displaystyle \cA \in \O_m$, $\displaystyle \cB \in L(\Rm, \Rm)$,
    $\displaystyle \norm{\cA - \cB} < \frac{1}{\norm{\cA^{-1}}}$, тогда

    \begin{itemize}
        \item[i)] $\cB \in \O_m$
        \item[ii)] $\displaystyle \norm{\cB^{-1}} \leqslant \frac{1}{\norm{\cA^{-
        1}}^{-1} - \norm{\cA - \cB}}$
        \item[iii)] $\displaystyle \norm{\cA^{-1} - \cB^{-1}} \leqslant
        \frac{\norm{\cA^{-1}}}{\norm{\cA^{-1}}^{-1} - \norm{\cA - \cB}}\norm{\cA
        - \cB}$
    \end{itemize}
\end{theorem}
\begin{proof}
    \item[i, ii)]
\[
    \norm{\cB \x} \geqslant \norm{\cA \x} - \norm{(\cA - \cB)\x} \geqslant
    \left( \frac{1}{\norm{\cA^{-1}}} - \norm{\cA - \cB} \right)\norm{\x}
\]
    первое неравенство --- неравенство треугольника, а второе выполнено потому,
    что
\[
    \norm{\x} = \norm{\cA \cA^{-1} \x} \leqslant \norm{\cA} \norm{\cA^{-1} \x}
\]
    Далее по лемме получаем обратимость $\cB$ и оценку на его норму.
    \item[iii)]
    \begin{gather*}
        \cA^{-1} - \cB^{-1} = \cA^{-1}(\cB - \cA)\cB^{-1} \\
        \norm{\cA^{-1} - \cB^{-1}} \leqslant \norm{\cA^{-1}} \norm{\cB - \cA}
        \norm{\cB^{-1}} \leqslant_{\text{i)}} \frac{\norm{\cA^{-1}}}{\norm{\cA^{-
        1}}^{-1} - \norm{\cA - \cB}}\norm{\cA - \cB}
    \end{gather*}
\end{proof}

\begin{corollary}
    Множество $\O_m$ открыто в метрической топологии $\langle L(\Rm, \Rm),
    \norm{\phantom{x }} \rangle$
\end{corollary}
