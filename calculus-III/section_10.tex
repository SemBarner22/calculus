\chapter{Интеграл}

\section{Измеримые функции}

\begin{definition}
	\textit{Разбиением} множества $E$ называется дизъюнктный набор множеств
	$e_i$ такой, что $E = \bigsqcup{e_i}$.
\end{definition}

\begin{definition}
	Функция $f\colon X \to \R$ называется \textit{ступенчатой}, 
	если существует конечное разбиение $X = \bigsqcup{e_i}$ (в контексте 
	мер множества $e_i$ должны быть измеримыми, то есть $e_i \in \cA$) такое,
	что на элементах разбиения $f$ постоянно:
\[
	f\big|_{e_i} = c_i
\]
	Разбиение $e_i$ в таком случае называется \textit{допустимым}.
\end{definition}

\begin{definition}
	\textit{Характеристической функцией} множества $E$ называется 
	фукнция
\begin{align*}
	\chi_E \colon X &\to \R \\
	x &\mapsto \begin{cases}
		1,~ x \in E \\
		0,~ x \notin E
	\end{cases}
\end{align*}
\end{definition}

\begin{remark}
	Если множество $E$ измеримо, то его характеристическая функция 
	является ступенчатой. Подойдет разбиение $X = E \bigsqcup \overline{E}$.
\end{remark}

\begin{remark}
	Ступенчатую функцию можно представить в виде:
\[
	f(x) = \sum_{i}{\chi_{e_i}(x) \cdot c_i}
\]
\end{remark}

\begin{lemma}(Свойства ступенчатых функций)
	\begin{itemize}
		\item Пусть $f$, $g$ ступенчатые. Тогда существует 
			общее допустимое разбиение.
		\item $f$, $g$ ступенчатые, тогда $\forall \a \in \R$ функции 
			$f + g$, $f \cdot g$, $\a f$, $\max(f, g)$, $|f|$, $\displaystyle
			\frac{f}{g}$ (при $g \neq 0$) ступенчатые.
	\end{itemize}
\end{lemma}
\begin{proof}
	\enewline
	\begin{itemize}
		\item Пусть $a_i$ и $b_i$ допустимые разбиения $f$ и $g$ соответственно. 
			Тогда в качестве их общего допустимого разбиения, очевидно,
			подойдет разбиение $a_i \cap b_j$.
		\item Очевидно. Если фукнция составлена из двух, можно рассмотреть 
			общее допустимое разбиение.
	\end{itemize}
\end{proof}

\textit{Контекст: $\langle X, \cA, \mu \rangle$ --- пространство с мерой.}

\begin{definition}
	Пусть $f \colon E \subseteq X \to \Rbar$. Тогда \textit
	{лебеговскими множествами} называются множества вида
\[
	E(f < a) \defeq \{\, x \in E \mid f(x) < a \,\}
\]
	Где $a \in \R$.
\end{definition}

\begin{remark}
	\enewline
	\begin{itemize}
		\item $E(f < a) = \overline{E(f \geqslant a)}$.
		\item $\displaystyle E(f \leqslant a) = \bigcap_{b > a}{E(f < b)} = 
			\bigcap_{n \in \bN}{E\left(f < a + \frac{1}{n}\right)}$
	\end{itemize}
\end{remark}

\begin{definition}
	$f \colon X \to \Rbar$, называется \textit{измеримой на множестве} $E \in \cA$,
	если
\[
	\forall a \in \R~~ E(f < a) \in \cA
\]
\end{definition}

\begin{definition}
	$f$ называется \textit{измеримой}, если она измерима на множестве $X$.
\end{definition}

\begin{definition}
	$f$ называется \textit{измеримой по Лебегу}, если она измерима в 
	контексте $\langle \Rm, \mathfrak{M}^m, \l_m \rangle$.
\end{definition}

\begin{remark}
	В определении измеримости на множестве можно брать 
	лебеговские множества любого вида (см. замечания выше:
	все эти множества измеримы или нет одновременно).
\end{remark}

\begin{lemma}(Измеримость непрерывных функций)
	Пусть $f \colon \Rm \to \R$ непрерывна. Тогда $f$ измерима по Лебегу.
\end{lemma}
\begin{proof}
	Имеем
\[
	E(f < a) = f^{-1}((-\infty, a))
\]
	Последнее множество открыто по определению непрерывности, а значит,
	измеримо.
\end{proof}

\begin{theorem}(Свойства измеримых функций)

	Пусть $f$ измерима на $E$. Тогда 
	\begin{itemize}
		\item $\forall a \in \R~ E(f = a)$ измеримо.
		\item $\forall \a > 0~ \a \cdot f$, $-f$ измеримы.
		\item $f$ измерима на $E_1, \ldots$ тогда $f$ измерима на $E' = \bigcup{E_i}$.
		\item $E' \subseteq E$, $E'$ измеримо, тогда $f$ измерима на $E'$.
		\item $f \neq 0$, тогда $\displaystyle \frac{1}{f}$ измерима на $E$.
		\item $f \geqslant 0,~\a > 0$, тогда $f^\a$ измерима на $E$.
	\end{itemize}
\end{theorem}
\begin{proof}
	Для доказательства всех пунктов просто преобразуем лебеговские 
	множества так, чтобы все стало очевидно из определения:
	\begin{itemize}
		\item $E(f = a) = E(f \leqslant a) \cap E(f \geqslant a)$.
		\item $E(\a \cdot f < a) = E\left(f < \frac{a}{\a}\right)$.
		\item $E'(f < a) = \bigcup{E_i(f < a)}$.
		\item $E'(f < a) = E' \cap E(f < a)$.
		\item
\begin{align*}
	E\left(\frac{1}{f} < a\right) = \left(E\left(\frac{1}{f} < a\right) 
	\cap E(f > 0)\right) 									
	\cup \left(E\left(\frac{1}{f} < a\right) \cap E(f < 0)\right) \\
	= \left(E\left(f > \frac{1}{a}\right) \cap E(f > 0)\right) 
	\cap \left(E\left(f < \frac{1}{a}\right) \cap E(f < 0)\right)
\end{align*}
		\item Аналогично.
	\end{itemize}
\end{proof}

\begin{theorem}(Измеримость пределов и супремумов)
	Пусть $f_n$ измеримы на $E$, тогда
	\begin{itemize}
		\item $\displaystyle \sup_{n}{f_n(x)}$, $\displaystyle \inf_{n}{f_n(x)}$
			измеримы на $E$.
		\item $\displaystyle \varlimsup_{n \to +\infty}{f_n(x)}$, 
			$\displaystyle \varliminf_{n \to +\infty}{f_n(x)}$ измеримы на $E$.
		\item Если $\displaystyle \forall x~ \exists f(x) =
			\lim_{n \to +\infty}{f_n(x)}$, то $f$ измерима на $E$.
	\end{itemize}
\end{theorem}
\begin{proof}
	\enewline
	\begin{itemize}
		\item Пусть $\displaystyle g(x) = \sup_{n}{f_n(x)}$. Рассмотрим лебеговы 
			множества $g$:
\[
	E(g > a) = \bigcup_{n}{E(f_n > a)}
\]
	Докажем это равенство подробно:
		\begin{itemize}
			\item[$\subseteq$]
				Пусть $y \in E(g > a)$. Тогда $\exists x \in E\colon g(x) > y$.
\[
	y < g(x) = \sup_{n}{f_n(x)} \Lra \exists 
\]
			\item[$\supseteq$]
		\end{itemize}
	\end{itemize}
\end{proof}
