\section{Формула Тейлора}

\begin{definition}
    Пусть $f \colon \O \subseteq \Rm \to \Rn$, $\O$ --- область, $i_1, \ldots,
    i_k \in \{\, 1, 2, \ldots, m \,\}$. Определим частные производные высшего
    порядка по индукции:
\[
    \pderi{f}{i_1, \ldots, i_k} \defeq \pderi{(\pderi{f}{i_1, \ldots, i_{k-
    1}})}{i_k}
\]
\end{definition}

\begin{theorem}(О независимости ч.п. от порядка дифференцирования)

    Пусть $f \colon \O \subseteq \R^2 \to \R$, $\O$ --- область, $(x_0, y_0) \in
    \O$, $\exists B((x_0, y_0), r) \subseteq \O$, причем в $B((x_0, y_0), r)$
    существуют $\pderi{f}{12}$ и $\pderi{f}{21}$, непрерывные в точке $(x_0,
    y_0)$. Тогда $\dder{f}{x_0, y_0}{12} = \dder{f}{x_0, y_0}{21}$
\end{theorem}
\begin{proof}
\[
        \a(h) = f(x_0 + h, y_0 + k) - f(x_0 + h, y_0) - f(x_0, y_0 + k) + f(x_0,
        y_0)
\]
    Тогда $\a(0) = 0$:
\begin{align*}
        \a(h) = \a(h) - \a(0) &\underset{\text{Лагранж}}{=}
        \a'(\tilde{h})h = [f'_x(x_0 + \tilde{h}, y_0 + k) - f'_x(x_0 + \tilde{h},
        y_0)]h \\ &\underset{\text{Лагранж}}{=} f''_{xy}(x_0 + \tilde{h}, y_0 +
        \tilde{k})hk
\end{align*}
    Аналогично введем $\b(k)$:
\[
        \b(k) = f(x_0 + h, y_0 + k) - f(x_0 + h, y_0) - f(x_0, y_0 + k) + f(x_0,
        y_0)
\]
    Тогда
\begin{align*}
        \b(k) = \b(k) - \b(0) &\underset{\text{Лагранж}}{=}
        \b'(\bar{k})k = [f'_y(x_0 + h, y_0 + \bar{k}) - f'_y(x_0, y_0 + \bar{k})]k
        \\ &\underset{\text{Лагранж}}{=} f''_{yx}(x_0 + \bar{h}, y_0 + \bar{k})hk
\end{align*}
    Заметим, что $\a(h) = \b(k)$. Осталось перейти к пределу при $(h, k) \to (0,
    0)$ и воспользоваться непрерывностью частных производных в точке $(x_0, y_0)$.
\end{proof}

\begin{corollary}

    Пусть $f \colon \O \subseteq \Rm \to \Rn$, $i_1, \ldots, i_k \in \{\, 1, 2,
    \ldots, m \,\}$, $\x \in \O$, $\exists B(\x, r) \subseteq \O$, причем
    в $B(\x, r)$ для любой перестановки индексов $\pi \in S_k$ существуют и
    непрерывны в $\x$ частные производные $\pderi{f}{i_{\pi_1}, \ldots,
    i_{\pi_k}}$. Тогда все они совпадают в точке $\x$.
\end{corollary}
\begin{proof}
    Доказательство сводится к координатным функциям, поэтому считаем, что $n = 1$.
    Предыдущая теорема дает возможность менять местами пары индексов. Осталось
    заметить, что группа перестановок порождается транспозициями.
\end{proof}

\begin{definition}
    Множество функций $f \colon \O \subseteq \Rm \to \Rn$, у которых все частные
    производные порядка не более $r$ существуют и непрерывны на $\O$, будем
    обозначать $C^r(\O)$
\end{definition}

\begin{definition}
    Пусть $k_1, k_2, \ldots, k_m \in \mathbb{N}_0$, тогда набор
    $k = (k_1, k_2, \ldots, k_m)$ будем называть \textit{мультииндексом}.
    Используются обозначения $|k| = k_1 + \ldots + k_m$,
\[
    \frac{\partial^k}{\partial x^k}f \defeq
    \frac{\partial^{|k|}}{\partial x_1^{k_1} \ldots \partial x_m^{k_m}}f
\]
\end{definition}

\begin{lemma}(Полиномиальная формула)
\[
    (a_1 + \ldots + a_m)^r = \sum_{n_1 = 1}^{m}{\sum_{n_2 = 1}^{m}{\ldots
    \sum_{n_r = 1}^{m}{a_{n_1} a_{n_2} \ldots a_{n_r}}}} =
    \sum_{|k| = r}{\frac{r!}{k_1! \ldots k_m!} a_1^{k_1} a_2^{k_2} \ldots
    a_m^{k_m}}
\]
\end{lemma}
\begin{proof}
    Первое равенство очевидно по правилам раскрытия скобок. Докажем второе
    равенство индукцией по $r$.

    \begin{itemize}
        \item Для $r = 1$ утверждение очевидно.
        \item Переход:
\begin{align*}
    &(a_1 + \ldots + a_m)^{r+1} = (a_1 + \ldots + a_m) \cdot (a_1 + \ldots +
    a_m)^r \\
    &= (a_1 + \ldots + a_m) \cdot \sum_{|k| = r}{\frac{r!}{k_1! \ldots k_m!}
    a_1^{k_1} a_2^{k_2} \ldots a_m^{k_m}} \\
    &= \sum_{|k| = r}{\frac{r!}{k_1! \ldots k_m!}
    a_1^{k_1 + 1} a_2^{k_2} \ldots a_m^{k_m}} + \ldots +
    \sum_{|k| = r}{\frac{r!}{k_1! \ldots k_m!}
    a_1^{k_1} a_2^{k_2} \ldots a_m^{k_m + 1}} \\
    &= [\text{переобозначим } k_i = k_i + 1 \text{ в } i \text{-й сумме}] \\
    &= \sum_{\substack{|k| = r + 1 \\ k_1 \geqslant 1}}{\frac{r! \cdot k_1}{k_1!
    \ldots k_m!} a_1^{k_1} a_2^{k_2} \ldots a_m^{k_m}} + \ldots +
    \sum_{\substack{|k| = r + 1 \\ k_m \geqslant 1}}{\frac{r! \cdot k_m}{k_1!
    \ldots k_m!} a_1^{k_1} a_2^{k_2} \ldots a_m^{k_m}} \\
    &= [\text{добавим все пропущенные слагаемые с } k_i = 0]
    \\ &= \sum_{|k| = r + 1}{\frac{r! \cdot k_1}{k_1!
    \ldots k_m!} a_1^{k_1} a_2^{k_2} \ldots a_m^{k_m}} + \ldots +
    \sum_{|k| = r + 1}{\frac{r! \cdot k_m}{k_1!
    \ldots k_m!} a_1^{k_1} a_2^{k_2} \ldots a_m^{k_m}} \\
    &= \sum_{|k| = r + 1}{\frac{r! \cdot (k_1 + k_2 \ldots + k_m)}{k_1!
    \ldots k_m!} a_1^{k_1} a_2^{k_2} \ldots a_m^{k_m}} \\
    &=\sum_{|k| = r + 1}{\frac{(r + 1)!}{k_1!
    \ldots k_m!} a_1^{k_1} a_2^{k_2} \ldots a_m^{k_m}}
\end{align*}
    \end{itemize}
\end{proof}

\begin{lemma}(О дифференцировании сдвига)

    $f \colon \O \subseteq \Rm \to \R$, $\O$ --- область, $f \in C^r(\O)$, $\ea
    \in \O$, $\h \in \Rm$, $\forall t \in [-1, 1]~~ \ea + t\h \in \O$, тогда
    для отображения $\f(t) = f(\ea + t\h)$ и для $k \leqslant r$ выполнено
\[
    \f^{(k)}(0) = \sum_{|j| = k}{\frac{k!}{j!}\h^j \frac{\partial^j f}{\partial
    x^j}(\ea)}
\]
\end{lemma}
\begin{proof}
    Для доказательства этого факта достаточно показать, что
\[
    \f^{(k)}(t) = \left(\frac{\partial}{\partial x_1}\h_1 + \ldots +
    \frac{\partial}{\partial x_m}\h_m\right)^k \cdot f(\ea + t\h)
\]
    Докажем по индукции:
    \begin{itemize}
        \item Для $k = 0$ утверждение очевидно.
        \item Переход:
\begin{align*}
        \f^{(k)}(t) &= (\f^{(k - 1)}(t))' =
        \left(\left(\frac{\partial}{\partial x_1}\h_1 + \ldots +
        \frac{\partial}{\partial x_m}\h_m\right)^{k - 1} f(\ea +
        t\h)\right)' \\
        &= \left(\frac{\partial}{\partial x_1}\h_1 + \ldots +
        \frac{\partial}{\partial x_m}\h_m\right) \cdot
        \left(\frac{\partial}{\partial x_1}\h_1 + \ldots +
        \frac{\partial}{\partial x_m}\h_m\right)^{k - 1} f(\ea +
        t\h) \\
        &= \left(\frac{\partial}{\partial x_1}\h_1 + \ldots +
        \frac{\partial}{\partial x_m}\h_m\right)^{k} f(\ea +
        t\h)
\end{align*}
    \end{itemize}
\end{proof}

\begin{theorem}(Формула Тейлора в форме Лагранжа)

    $f \colon \O \subseteq \Rm \to \R$, $f \in C^{r+1}(\O)$, $\O$ --- область,
    $\ea \in \O$, $\x \in B(\x, r) \subseteq \O$, тогда
\[
    \exists \theta \in (0, 1)\colon~
    f(\x) = \sum_{|k| \leqslant r}{\frac{1}{k!}
    \frac{\partial^k f}{\partial x^k}(\ea) (\x - \ea)^k}
    + \sum_{|k| = r+1}{\frac{1}{k!} \frac{\partial^k f}{\partial x^k}(\ea +
    \theta(\x - \ea))(\x - \ea)^k}
\]
\end{theorem}
\begin{proof}
    Пусть $\f(t) = f(\ea + t\h)$ для $\x = \ea + \h$, $\h = \x - \ea$.
    Выпишем формулу Тейлора в форме Лагранжа для $\f$:
\[
    \f(1) = \sum_{k = 0}^{r}{\frac{\f^{(k)}(0)}{k!}} +
    \frac{\f^{(r+1)}(\theta)}{(r+1)!}
\]
    Пользуясь леммой о дифференцировании сдвига, получаем искомое равенство.
\end{proof}

\begin{theorem}(Формула Тейлора в форме Пеано)

    $f \colon \O \subseteq \Rm \to \R$, $f \in C^{r+1}(\O)$, $\O$ --- область,
    $\ea \in \O$, $\x \in B(\x, r) \subseteq \O$, тогда
\[
    \exists \theta \in (0, 1)\colon~
    f(\x) = \sum_{|k| \leqslant r}{\frac{1}{k!}
    \frac{\partial^k f}{\partial x^k}(\ea) (\x - \ea)^k}
    + o(\norm{\x - \ea}^r)
\]
\end{theorem}
\begin{proof}
    Достаточно показать, что
\[
    \sum_{|k| = r+1}{\frac{1}{k!} \frac{\partial^k f}{\partial x^k}(\ea +
    \theta(\x - \ea))(\x - \ea)^k} = o(\norm{\x - \ea}^r)
\]
    Проверим это:
\begin{align*}
    &\sum_{|k| = r+1}{\frac{1}{k!} \frac{\partial^k f}{\partial x^k}(\ea +
    \theta(\x - \ea)) \frac{(\x - \ea)^k}{\norm{\x - \ea}^r}}
    = \sum_{|k| = r+1}{\frac{1}{k!} \frac{\partial^k f}{\partial x^k}(\ea +
    \theta(\x - \ea)) \frac{(\x_1 - \ea_1)^{k_1} \ldots (\x_m -
    \ea_m)^{k_m}}{\norm{\x - \ea}^r}} \\
    &= \sum_{|k| = r+1}{\frac{1}{k!} \frac{\partial^k f}{\partial x^k}(\ea +
    \theta(\x - \ea)) \frac{(\x_1 - \ea_1)^{k_1}}{\norm{\x - \ea}^{k_1}} \ldots
    \frac{(\x_m - \ea_m)^{k_m}}{\norm{\x - \ea}^{k_m}}} \norm{\x - \ea}
\end{align*}
    Все дроби вида
\[
    \frac{(\x_i - \ea_i)^{k_i}}{\norm{\x - \ea}^{k_i}}
\]
    меньше единицы, выражения вида
\[
    \frac{1}{k!} \frac{\partial^k f}{\partial x^k}(\ea + \theta(\x - \ea))
\]
    постоянны. Поэтому
\[
    \sum_{|k| = r+1}{\frac{1}{k!} \frac{\partial^k f}{\partial x^k}(\ea +
    \theta(\x - \ea)) \frac{(\x_1 - \ea_1)^{k_1}}{\norm{\x - \ea}^{k_1}}
    \ldots \frac{(\x_m - \ea_m)^{k_m}}{\norm{\x - \ea}^{k_m}}} \norm{\x -
    \ea} \to 0
\]
    При $\norm{\x - \ea} \to 0$.
\end{proof}
