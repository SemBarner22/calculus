\chapter{Теория меры}

\section{Системы множеств}

\begin{definition}
    \textit{Полукольцом подмножеств} множества $X$ называют $\cP \subseteq 2^X$,
    удовлетворяющее условиям
    \begin{itemize}
        \item[1.] $\varnothing \in \cP$.
        \item[2.] $A, B \in \cP \Lra A \cap B \in \cP$.
        \item[3.] $\displaystyle \forall A, B \in \cP~ \exists B_1, \ldots,
        B_k \in \cP\colon~ A \setminus B = \bigsqcup_{i = 1}^{k}{B_i}$.
    \end{itemize}
\end{definition}

\begin{definition}
    \textit{Ячейкой} в $\Rm$ называется множество вида
\[
    [\ea, \eb) = \{\, \x \in \Rm \mid \ea_i \leqslant \x_i < \eb_i \,\}
\]
\end{definition}

\begin{theorem}(Свойства полуколец)
    \begin{itemize}
        \item[1.] $A \in \cP \nRightarrow \bar{A} \in \cP$
        \item[2.] $A, A' \in \cP \nRightarrow A~@~A' \in \cP$, $@ \in
        \{\, \cup, \setminus, \triangle \,\}$
        \item[3.] $\displaystyle A_1, \ldots, A_n \in \cP \Lra A \setminus
        \left(\bigcup_{i = 1}^{n}{A_i}\right) = \bigsqcup_{fin}{D_j}$
    \end{itemize}
\end{theorem}

\begin{definition}
    \textit{Алгеброй подмножеств} множества $X$ называется множество $\cA \in
    2^X$ такое, что выполнены аксиомы:
    \begin{itemize}
        \item[1.] $X \in \cA$
        \item[2.] $A, B \in \cA \Lra A \setminus B \in \cA$
    \end{itemize}
\end{definition}

\begin{theorem}(Свойства алгебр)
    \begin{itemize}
        \item[1.] $\varnothing = X \setminus X \in \cA$
        \item[2.] $A \cap B = A \setminus (A \setminus B) \in \cA$
        \item[3.] $\overline{A} = X \setminus A \in \cA$
        \item[4.] $A \cup B \in \cA$
\[
        X \setminus (A \cup B) = (X \setminus A) \cap (X \setminus B)
\]
        \item[5.] $\displaystyle A_1, \ldots A_n \in \cA \Lra \bigcup_{i = 1}^{n}{A_i} \in
        \cA$, $\displaystyle \bigcap_{i = 1}^{n}{A_i} \in \cA$
        \item[6.] Алгебра подмножеств является полукольцом подмножеств
    \end{itemize}
\end{theorem}

\begin{definition}
    \textit{$\sigma$-Алгеброй подмножеств} множества $X$ называется алгебра
    подмножеств $\cA$, удовлетворяющая дополнительной аксиоме: \\
    $\displaystyle \{\,A_n\,\} \in \cA \Lra \bigcup_{n = 1}^{+\infty}{A_n} \in \cA$.
\end{definition}

\begin{lemma}(О нарезке)

    Пусть $A_0, A_1, \ldots, A_n \subseteq X$. Тогда набор множеств
\[
    B_1 = A_1,~ B_2 = A_2 \setminus A_1, \ldots, B_k = A_k \setminus
        \left (\bigcup_{i = 1}^{k - 1}{A_i} \right), \ldots
\]
    дизъюнктен, причем
\[
    \bigsqcup_{i = 1}^{n}{B_i} = \bigcup_{i = 1}^{n}{A_i}
\]
\end{lemma}

\begin{lemma}(О минимальной алгебре)

    Пусть $\cP$ --- полукольцо. Положим $\cA_0$ --- система подмножеств,
    состоящая из всевозможных конечных объединений множеств из $\cP$, а так
    же из их дополнений. Тогда
    \begin{itemize}
        \item $\cA_0$ --- алгебра подмножеств.
        \item Для любой алгебры $\cA \supseteq \cP$ верно, что $\cA \supseteq \cA_0$
    \end{itemize}
\end{lemma}

\section{Объём}

\begin{definition}
    Пусть $\cP$ --- полукольцо, $\mu \colon \cP \to \Rbar$ называется
    \textit{конечно-аддитивной}, если
    \begin{itemize}
        \item[1.] $\mu$ принимает не более одного значения из $\{\,+\infty, -\infty\,\}$
        \item[2.] $\mu(\varnothing) = 0$
        \item[3.] $A_1, \ldots, A_n \in \cP$, $A_i \cap A_{j \neq i} = \varnothing$,
        тогда если оказалось, что $\displaystyle A = \bigsqcup_{i = 1}^{n}{A_i} \in \cP$,
        то \\ $\displaystyle \mu(A) = \sum_{i = 1}^{n}{\mu(A_i)}$
    \end{itemize}
\end{definition}

\begin{definition}
    Пусть $\mu \colon \cP \to \Rbar$ называется \textit{объёмом}, если
    \begin{itemize}
        \item[1.] $\mu$ конечно-аддитивна
        \item[2.] $\mu \geqslant 0$
    \end{itemize}
\end{definition}

\begin{definition}
    Объём называется \textit{конечным}, если $\mu(X) < +\infty$.
\end{definition}

\begin{definition}
    \textit{Классическим объёмом} в $\Rm$ называется объём, заданный на
    полукольце ячеек в $\Rm$, вычисляющийся по формуле
    $\displaystyle \mu([\ea, \eb)) = \prod_{k = 1}^{m}{(\eb_k - \ea_k)}$.
\end{definition}

\begin{lemma}(Монотонность объёма)

    Для объёма $\mu,$ $A, B \in \cP$, $A \subseteq B$ выполено $\mu(A) \leqslant \mu(B)$.
\end{lemma}
\begin{proof}
\[
    B = A + B \setminus A = A + \bigsqcup{D_i} \Lra \mu(B) = \mu(A) + \sum{\mu(D_i)}
    \geqslant \mu(A)
\]
\end{proof}

\begin{theorem}(Свойства объёма)
    \begin{itemize}
        \item[1.] $\displaystyle \forall A, \text{ дизъюнктных } A_1 \ldots, A_n \in \cP
        \colon~ \bigsqcup_{i = 1}^{n}{A_i} \subseteq A \Lra
        \sum_{i = 1}^{n}{\mu(A_i)} \leqslant \mu(A)$ \\ (усиленная монотонность)
        \item[2.] $\displaystyle \forall A, A_1 \ldots, A_n \in \cP
        \colon~ A \subseteq \bigcup_{i = 1}^{n}{A_i} \Lra
        \mu(A) \leqslant \sum_{i = 1}^{n}{\mu(A_i)}$ \\ (конечная полуаддитивность)
        \item[3.] $A, B, A \setminus B \in \cP \Lra \mu(A \setminus B) \geqslant
                \mu(A) - \mu(B)$
    \end{itemize}
\end{theorem}
\begin{proof}
    \enewline
    \begin{itemize}
        \item[1.]
\begin{align*}
    &A \setminus \bigsqcup_{i = 1}^{n}{A_i} = \bigsqcup_{i = 1}^{k}{D_i} \Lra A
    = \left(\bigsqcup_{i = 1}^{n}{A_i}\right) \sqcup \left(\bigsqcup_{i = 1}^{k}{D_i}\right) \\
    &\Lra \mu(A) = \sum_{i = 1}^{n}{\mu(A_i)}) + \sum_{i = 1}^{k}{\mu(D_i)} \geqslant
    \sum_{i = 1}^{n}{\mu(A_i)}
\end{align*}
        \item[2.] Сейчас будет использован стандартный прием, смысл которого заключается в
                переходе от простого объединения к дизъюнктному. Пусть $B_k = A \cap A_k$.
                Тогда $A = \bigcup{B_k}$. Теперь \textit{нарежем $B_k$}:
\[
    C_k = B_k \setminus \bigcup_{i = 1}^{k - 1}{B_i}
\]
    При $k > 1$ и $C_1 = B_1$. Набор $C_k$ получился дизъюнктным:
\[
    A = \bigsqcup_{i = 1}^{n}{C_i}
\]
	Сами множетсва $C_k$ могут и не быть в $\cP$, но $B_k \in \cP$ как пересечения
	множеств из $\cP$. Из определения $C_k$ имеем, что,
\[
	C_k = \bigsqcup_{i = 1}^{j_k}{D_{ki}}
\]
	Тогда можно вычислить объём $A$:
\[
	\mu(A) = \sum_{k, i}{\mu(D_{ki})}
\]
	Теперь воспользуемся монотонностью объёма: $C_k \subseteq B_k \subseteq A_k$:
\[
	\sum_{j}{D_{kj}} \leqslant \mu(B_k) \leqslant \mu(A_k)
\]
	Поэтому
\[
	\mu(A) = \sum_{k}{\sum_{j}{D_{kj}}} \leqslant \sum_{k}{\mu(A_k)}
\]
		\item[3.]
				\enewline
				\begin{itemize}
					\item[(a)]
\[
	B \subseteq A \Lra \mu(A) = \mu(A \setminus B) + \mu(B) \Lra \mu(A \setminus B)
	= \mu(A) - \mu(B)
\]
					\item[(b)]
\[
	A \setminus B = A \setminus (A \cap B) \Lra \mu(A \setminus B) = \mu(A) - \mu(A \cap B)
	\geqslant \mu(A) - \mu(B)
\]
				\end{itemize}
    \end{itemize}
\end{proof}

\section{Мера}

\begin{definition}
    \textit{Мерой} называется объём $\mu \colon \cP \to \Rbar$, обладающий свойством
    счётной аддитивности.
\end{definition}

\begin{theorem}
    Пусть $\mu \colon \cP \to \Rbar$ --- объём. Тогда эквивалентны утверждения:
    \begin{itemize}
        \item[1.] $\mu$ счетно-аддитивен
        \item[2.] $\mu$ счетно-полуаддитивен
    \end{itemize}
\end{theorem}
\begin{proof}
	Импликация $1 \Lra 2$ доказывается практически так же, как второй
	пункт предыдущей теоремы (используется нарезка). Докажем $2 \Lra 1$.
	Для этого воспользуемся усиленной монотонностью объёма ($A_i$ дизъюнктны):
\[
	\forall N~ \sum_{i = 1}^{N}{\mu(A_i)} \leqslant \mu(A)
\]
	Добавляя к этому посылку:
\[
	\forall N~ \sum_{i = 1}^{N}{\mu(A_i)} \leqslant \mu(A) \leqslant
	\sum_{i = 1}^{+\infty}{\mu(A_i)}
\]
	И переходя к пределу при $N \to +\infty$, получаем требуемое.
\end{proof}

\begin{theorem}
    Пусть $\cA$ --- алгебра, $\mu \colon \cA \to \Rbar$ --- объём. Тогда
    эквивалентны утверждения:
    \begin{itemize}
        \item[1.] $\mu$ счетно-аддитивно
        \item[2.] $\mu$ \textit{непрерывно снизу}, то есть
        $\displaystyle A,~A_1, A_2 \ldots~ \in \cA\colon A_1 \subset A_2 \subset \ldots$;
        $\displaystyle A = \bigcup_{i = 1}^{+\infty}{A_i} \Lra \\ \mu(A) = \lim_{n \to
        +\infty}{\mu(A_i)}$
    \end{itemize}
\end{theorem}
\begin{proof}
	\enewline
	\begin{itemize}
		\item [$1 \Lra 2$] Нарежем множетсва:
\[
	B_1 = A_1, \ldots, B_k = A_k \setminus \left(\bigcup_{i = 1}^{k - 1}{A_i}\right)
\]
	Как всегда, $B_k$ дизъюнктны, причем $A = \bigsqcup_{i = 1}^{+\infty}{B_k}$.
	Тогда, пользуясь счетной аддитивностью, имеем:
\[
	\mu(A) = \sum_{i = 1}^{+\infty}{\mu(B_i)}
	= \lim_{N \to +\infty}{\sum_{i = 1}^{N}{\mu(B_i)}}
	= \lim_{N \to +\infty}{\mu(A_k)}
\]
		\item [$2 \Lra 1$] Пусть есть дизъюнктные $A_k$. Сделаем из них $C_k$:
\[
	C_k = \bigsqcup_{i = 1}^{k}{A_i}
\]
		Тогда $C_1 \subseteq C_2 \subseteq \ldots$. Воспользуемся непрерывностью снизу:
\[
	\mu(A) = \mu(C) = \lim_{N \to +\infty}{\mu(C_k)}
	= \lim_{k \to +\infty}{\sum_{i = 1}^{k}{\mu(A_k)}}
	= \sum_{k = 1}^{+\infty}{\mu(A_k)}
\]
	\end{itemize}
\end{proof}

\begin{theorem}
    Пусть $\cA$ --- алгебра, $\mu \colon \cA \to \Rbar$ --- \textbf{конечный} объём.
    Тогда эквивалентны утверждения:
    \begin{itemize}
        \item[1.] $\mu$ счетно-аддитивен
		\item[2.] $\mu$ \textit{непрерывно сверху}
\[
	A,~A_1, A_2 \ldots~ \in \cA\colon A_1 \supset A_2 \supset \ldots;~
    \displaystyle A = \bigcap_{i = 1}^{+\infty}{A_i} \Lra \\ \mu(A) = \lim_{n \to
    +\infty}{\mu(A_i)}
\]
        \item[3.] $\mu$ \textit{непрерывно сверху на пустом множестве}, то есть
			при условии, что $A = \varnothing$.
    \end{itemize}
\end{theorem}
\begin{proof}
	\enewline
	\begin{itemize}
		\item [$1 \Lra 2$] Будем пользоваться непрерывностью снизу, но для этого
			нужна подготовка:
\[
	B_1 = A_1 \setminus A, \ldots, B_k = A_1 \setminus A_k
\]
	Тогда $B_1 \subseteq B_2 \subseteq \ldots$; $B = \bigcup{B_k}$:
\[
	\mu(A_1) - \mu(A) = \mu(B) = \lim_{k \to +\infty}{\mu(B_k)}
	= \lim_{k \to +\infty}{\mu(A_1)} - \lim_{k \to +\infty}{\mu(A_k)}
\]
	Откуда
\[
	\mu(A) = \lim_{k \to +\infty}{\mu(A_k)}
\]
		\item[$2 \Lra 3$] Очевидно.
		\item[$3 \Lra 1$] Пусть $C_k$ дизъюнктны. Положим
\[
	A_k = \bigsqcup_{i = k+1}^{+\infty}{C_k}
\]
	Тогда $A_1 \supseteq A_2 \supseteq \ldots$; $A = \bigcap{A_k} = \varnothing$.
	Вообще говоря, $A_k \notin \cA$. Но в нашем случае
\[
	A_k = C \setminus \bigsqcup_{i = 1}^{k}{C_i} \in \cA
\]
	Далее надо как-то воспользоваться непрерывностью сверху:
\[
	C = A_k \sqcup \left(\bigsqcup_{i = 1}^{k}{C_k}\right)
	\Lra \mu(C) = \mu(A_k) + \sum_{i = 1}^{k}{\mu(C_k)}
\]
	Переходя к пределу при $k \to +\infty$, получаем требуемое.
	\end{itemize}
\end{proof}

\section{О стандартном продолжении меры}

\begin{definition}
    \textit{Пространством с мерой} называется тройка $\langle X, \cA, \mu \rangle$,
    где $\cA$ --- $\sigma$-алгебра, $\mu \colon \cA \to \Rbar$ --- мера.
\end{definition}

\begin{definition}
    $\langle X, \cA, \mu \rangle$ называется \textit{полным} (соответственно мера
    называется \textit{полной}), если $\forall E \in \cA \colon~ \mu(E) = 0 \Lra
    \forall A \subseteq E~ A \in \cA$ и $\mu(A) = 0$.
\end{definition}

\begin{definition}
    $\langle X, \cP, \mu \rangle$ называется \textit{$\sigma$-конечным} (соответственно
    мера называется \textit{$\sigma$-конечной}), если $\displaystyle
    X = \bigcup_{i = 1}^{+\infty}{B_k}$, где $\mu(B_k) < +\infty$.
\end{definition}

\begin{theorem}(О стандартном продолжении меры)

    $\langle X, \cP, \mu_0 \rangle$, $\mu_0$ --- $\sigma$-конечный объём. Тогда
    $\exists~ \sigma$-алгебра $\cA$ и мера $\mu \colon \cA \to \Rbar\colon$
    \begin{itemize}
        \item[1.] $\cP \subseteq \cA$, $\mu\big|_{\cP} = \mu_0$
        \item[2.] $\mu$ полная
        \item[3.] Если $\cA' \supseteq \cP$, $\mu'\big|_{\cP} = \mu_0$,
        $\mu'$ --- полная, тогда $\cA \subseteq \cA'$ и $\mu'\big|_{\cA} = \mu$
        \item[4.] Если $\cP'$ --- полукольцо, $\mu'$ --- мера на $\cP'$,
        $\cP \subseteq \cP' \subseteq \cA$, тогда $\mu' = \mu\big|_{\cP'}$
        \item[5.] $\displaystyle \forall A \in \cA~ \mu(A) =
        \inf{\left(\sum_{k = 1}^{+\infty}{\mu_0(P_k)} ~\bigg|~ A \subseteq \bigcup_{k = 1}
        ^{+\infty}{P_k}, P_k \in \cP\right)}$
    \end{itemize}
\end{theorem}

\section{Мера Лебега}

\begin{theorem}
    Классический объём в $\Rm$ является $\sigma$-конечной мерой.
\end{theorem}
\begin{proof}
	$\s$-конечность очевидна. Докажем счетную полуаддитивность. Пусть
	$[a, b) \subseteq \bigcup_{n = 1}^{+\infty}{[a_n, b_n)}$. Наша цель ---
	показать, что $\mu[a, b) \leqslant \sum_{n = 1}^{+\infty}{\mu[a_n, b_n)}$.
	Выберем $b' < b$ таким образом, чтобы
\[
	\mu[a, b) - \e < \mu[a, b') < \mu[a, b)
\]
	И $a_n' < a_n$ так, чтобы
\[
	\forall n~ \mu[a_n, b_n) < \mu[a_n', b_n) < \mu[a_n, b_n) + \frac{\e}{2^n}
\]
	Теперь отрезок $[a, b']$ накрыт счетным объедиением интервалов $(a'_n, b_n)$.
	Выберем из этого покрытия конечное:
\[
	[a, b'] \subseteq \bigcup_{n = 1}^{N}{(a'_n, b_n)} \subseteq
	\bigcup_{n = 1}^{N}{[a'_n, b_n)}
\]
	Воспользуемся конечной полуаддитивностью:
\[
	\mu[a, b') \leqslant \sum_{n = 1}^{N}{\mu[a'_n, b_n)} \leqslant
	\sum_{n = 1}^{N}{\left(\mu[a_n, b_n) + \frac{\e}{2^n}\right)} \leqslant
	\sum_{n = 1}^{N}{\mu[a_n, b_n)} + \e
\]
	С другой стороны:
\[
	\mu[a, b) - \e < \mu[a, b') \leqslant \sum_{n = 1}^{N}{\mu[a_n, b_n)} + \e
\]
	Откуда и следует требуемое.
\end{proof}

\begin{definition}
    \textit{Мерой Лебега} называется стандартное продолжение классического объёма.
\end{definition}

\begin{definition}
    Алгебра, на которой определена мера Лебега, обозначается $\mathfrak{M}$.
\end{definition}

\begin{definition}
    \textit{Измеримыми по Лебегу} называются множества $A \in \mathfrak{M}$.
\end{definition}

\begin{theorem}(Свойства меры Лебега)
    \begin{itemize}
        \item[1.] Объединения и пересечения измеримых множеств измеримы.
        \item[2.] Все открытые и замкнутые множества измеримы.
    \end{itemize}
\end{theorem}

\begin{lemma}(О структуре открытых множеств)
    \begin{itemize}
        \item[1.] $\Od \subseteq \Rm$ открыто $\Lra~ \exists Q_i$ --- ячейки в $\Rm$
        такие, что $\displaystyle \Od = \bigsqcup_{i}{Q_i}$, причем можно
        дополнительно считать, что выполнено что-либо из нижеперечисленного:
        \begin{itemize}
            \item[(a)] Ячейки имеют рациональные (двоично-рациональные) координаты
            \item[(b)] $\Cl(Q_i) \subseteq \Od$
            \item[(c)] $Q_i$ --- кубы
        \end{itemize}
        \item[2.] Пусть $E$ измеримо в $\Rm$, $\lambda(E) = 0$, тогда
        $\forall \e > 0~\exists Q_i$ --- ячейки в $\Rm$ такие, что
\[
		E \subseteq \bigcup_{i}{Q_i} \text{ и }
		\sum_{i}{\mu(Q_i)} < \e
\]
		причем можно потребовать свойства $(a) - (c)$ и дизъюнктность набора ячеек.
    \end{itemize}
\end{lemma}
\begin{proof}
	\enewline
	\begin{itemize}
		\item[1.] Всем точкам $\cO$ сопоставим ячейку требуемого типа с рациональными
			параметрами. Таких ячеек не более чем счетное множество:
\[
	\cO = \bigcup_{i = 1}^{+\infty}{\widetilde{Q}_i}
\]
		Теперь надо сделать эти ячейки дизъюнктыми. Для этого воспользуемся тем,
		что их счетное число:
\[
	Q'_1 = \widetilde{Q}_1, \ldots, Q'_k = \widetilde{Q}_k \setminus \bigcup_{i = 1}^{k - 1}
	{\widetilde{Q}_i}
\]
	$\forall i~ Q'_i$ выражается через дизъюнктное объединение ячеек нужного типа,
	так как полукольцо ячеек --- полукольцо. Тогда:
\[
	\cO = \bigcup_{i = 1}^{+\infty}{\widetilde{Q}_i}
	= \bigsqcup_{i = 1}^{+\infty}{Q'_i}
	= \bigsqcup_{i = 1}^{+\infty}{\bigsqcup_{j = 1}^{k_i}{D_{ij}}}
\]
		\item[2.] Пятый пункт теоремы о продолжении поставляет нам способ получить
			ячейки, удовлетворяющие всему, кроме дизъюнктности и $(a) - (c)$. Эти
			свойства придется добывать самостоятельно. Пусть множества $\widetilde{P}_i$
			пришли из теоремы для $\e = \frac{\e}{2}$:
\[
	\sum{\l(\widetilde{P}_i)} < \frac{\e}{2}
\]
	Немного расширим эти множества до открытых $P_i \supset \widetilde{P}_i$. Эти множества
	по первому пункту текущей леммы представляются в виде нужного нам объединения:
\[
	P_i = \bigsqcup{D_{ij}}
\]
	Потребуем (при выборе $P_i$):
\[
	\l(\widetilde{P}_i) < \l(P_i) < \l(\widetilde{P}_i) + \frac{\e}{2^i}
\]
	Тогда получается то, что нужно:
\[
	\sum{\l(D_{ij})} = \sum{\l(P_i)} < \sum{\left(\l(\widetilde{P}_i) + \frac{\e}{2^i}\right)}
	\leqslant \e
\]
	\end{itemize}
\end{proof}

\begin{theorem}(Свойства меры Лебега)
    \begin{itemize}
        \item[3.] Пример: канторово множество в $\R$. Строится итеративно:
\begin{align*}
	\mathfrak{K}_0 &= \left[0, 1\right] \\
	\mathfrak{K}_1 &= \left[1, \frac{1}{3}\right] \cup \left[\frac{2}{3}, 1\right]\\
	\mathfrak{K}_2 &= \left[1, \frac{1}{9}\right] \cup \left[\frac{2}{9}, \frac{3}{9}\right]
		\cup \left[\frac{6}{9}, \frac{7}{9}\right] \cup \left[\frac{8}{9}, 1\right]
			   \\	&\cdots \\
	\mathfrak{K} &= \bigcap_{i = 0}^{+\infty}{\mathfrak{K}_i}
\end{align*}
		Получившееся множество измеримо, потому что является пересечением измеримых
		множеств. Его мера равна нулю, потому что к нулю стремится мера
		$\mathfrak{K}_i$. С другой стороны, можно задать это же множество в другом виде:
\[
	\mathfrak{K} = \{\, 0.\e_1\e_2\ldots_3  \mid \e_i \in \{\,0,2\,\}  \,\}
\]
		По построению понятно, что из отрезка просто выкинули все точки, троичная запись
		которых содержит хотя бы одну единицу. Тогда $\mathfrak{K}$ континуально
		как множество бинарных последовательностей.
        \item[4.] Пример: неизмеримое множество. Совершенно очевидно, что одними
			только пересечениями и объединениями не получится построить неизмеримое множество.
			Привлечем к борьбе с измеримостью фактормножества. Пусть $\sim$ --- отношение,
			заданное на отрезке $[0, 1]$ такое, что:
\[
	a \sim b \Llra a - b \in \bQ
\]
		Рассмотрим тогда фактормножество $A = [0, 1] ~/ \sim$. Предположим, что оно
		измеримо. В одной из следующих теорем будет доказано, что мера Лебега инвариантна
		относительно сдвига, то есть $\l(A) = \l(A + q)$. Пусть тогда
\[
	B = \bigsqcup_{q \in [-1, 1] \cap \bQ}{\left(A + q\right)}
\]
		Дизъюнктность объединяемых множеств очевидна. Тогда $B$ тоже измеримо, причем
\[
	B \subseteq [-1, 2], B \supseteq [0, 1] \Lra 0 < \l(B) < +\infty
\]
		Но с другой стороны можно вычислить $\l(B)$ через $\l(A)$:
\[
	\l(B) = \sum_{q \in [-1, 1] \cap \bQ}{\l(A + q)} = \begin{cases}
		0,~ \l(A) = 0 \\
		+\infty,~ \l(A) > 0
	\end{cases}
\]
		Возникло противоречие, поэтому $A$ неизмеримо.
	    \item[5.] \begin{itemize}
                    \item $A$ ограничено, тогда $\lambda(A) < +\infty$
                    \item $A$ открыто, тогда $\lambda(A) > 0$
                    \item $\lambda(A) = 0 \Lra$ У $A$ нет внутренних точек
                  \end{itemize}
        \item[6.] $A$ измеримо, тогда $\forall \e > 0$
                \begin{itemize}
                    \item $\exists G_{\e}$ открытое такое, что
                    $A \subset G_{\e}$, $\lambda(G_{\e} \setminus A) < \e$
                    \item $\exists F_{\e}$ замкнутое такое, что
                    $F_{\e} \subset A$, $\lambda(A \setminus F_{\e}) < \e$
                \end{itemize}
    \end{itemize}
\end{theorem}
\begin{proof}
	\enewline
	\begin{itemize}
		\item[5.] Очевидно.
		\item[6.] \begin{itemize}
			\item
			\begin{itemize}
				\item[(i)] Пусть $\l(A) < +\infty$. Из последнего пункта теоремы о
					продолжении получаем набор $P_i$ такой, что
\[
	\l(A) < \sum{\l(P_i)} < \l(A) + \e
\]
			Далее немного расширим $P_i$ до открытого $\widetilde{P}_i$, которое
			наполовину замкнем до ячейки $\widetilde{\widetilde{P}}_i$:
\[
	P_i \subset \widetilde{P}_i \subset \widetilde{\widetilde{P}}_i \in \cP
\]
			Расширим мы несильно, а именно:
\[
	\l\left(\widetilde{\widetilde{P}}_i\right) - \l(P_i) < \frac{\e}{2^i}
\]
			В таком случае положим
\[
	G_{2\e} = \bigcup{\widetilde{P}_i}
\]
     		и проверим, что оно подходит:
\[
	\l(A) < \l(G_{2\e}) \leqslant \sum{\l\left(\widetilde{\widetilde{P}}_i\right)}
	\leqslant \l(A) + 2\e
\]
	$A \subset G_{2\e}$, поэтому:
\[
	\l(G_{2\e} \setminus A) = \l(G_{2\e}) - \l(A) < 2\e
\]
				\item[(ii)] $\l(A) = +\infty$. Воспользуемся $\s$-конечностью $\Rm$:
\[
	\Rm = \bigsqcup{Q_i} \Lra A = \bigsqcup{A \cap Q_i}
\]
			Пользуясь (i), расширим $A \cap Q_i$ до открытого множества $(G_{\e})_i$
			так, чтобы
\[
	\l((G_\e)_i \setminus (A \cap Q_i)) < \frac{\e}{2^i}
\]
			Пусть тогда $G_\e = \bigcup{(G_\e)_i}$. Проверим, что оно подходит:
\[
	\l(A) <	\l(G_\e) \leqslant \sum{\l((G_\e)_i)} < \sum{\l(A \cap Q_i)} + \e = \l(A) + \e
\]
			\end{itemize}
			\item Переход к дополнению: $A^c$ открыто, поэтому
\[
		\exists G_\e\colon~	\l(G_\e \setminus A^c) < \e \Lra \l(A \setminus G_\e^c)
		= \l(G_\e \setminus A^c) < \e
\]
		То есть $F_\e = G_\e^c$ подходит.
		\end{itemize}
	\end{itemize}
\end{proof}

\begin{definition}
    Пусть $\cA \subseteq 2^X$, тогда \textit{борелевской оболочкой} множества $\cA$
    называют минимальную по включению $\sigma$-алгебру, содержащую $\cA$.
\end{definition}

\begin{definition}
    \textit{Борелевской $\sigma$-алгеброй} называется борелевская оболочка
    всех открытых множеств.
\end{definition}

\begin{corollary}
    $A$ измеримо, тогда $\exists$ борелевские $B, C\colon$ $B \subset A \subset C$
    такие, что $\lambda(C \setminus B) = 0$.
\end{corollary}
\begin{proof}
	Возьмем в качестве $B = \bigcup{F_{\frac{1}{n}}}$, $C = \bigcap{G_{\frac{1}{n}}}$.
	Имеем
\[
	F_{\frac{1}{n}} \subseteq A \subseteq G_{\frac{1}{n}}
\]
	И
\[
	\forall n~ \l(B \setminus A) < \l(G_{\frac{1}{n}} \setminus F_{\frac{1}{n}}) < \frac{2}{n}
\]
	Поэтому $\l(B \setminus A) = 0$. $B$, $A$ борелевские, потому что являются пересечением 
	и объединеием борелевских множеств.
\end{proof}

\begin{corollary}
    $A$ измеримо, тогда $A = B \cup \mathfrak{N}$, $B$ --- борелевское,
    $\lambda(\mathfrak{N}) = 0$.
\end{corollary}
\begin{proof}
	Возьмем в качестве $B$ множество $B$ из предыдущего следствия, а в качестве 
	$\mathfrak{N}$ --- множество $A \setminus B$. Тогда $\l(\mathfrak{N}) \leqslant
	\l(C \setminus B) = 0$.
\end{proof}

\begin{corollary}(Регулярность меры Лебега)

    Пусть $A$ измеримо, тогда
\[
    \lambda(A) = \inf_{\substack{G \supset A \\ G \text{ открыто}}}{\lambda(G)}
    = \sup_{\substack{F \subset A \\ F \text{ замкнуто}}}{\lambda(F)}
    = \sup_{\substack{K \subset A \\ K \text{ компакт}}}{\lambda(K)}
\]
\end{corollary}
\begin{proof}
	Первые два пункта следуют из существования последовательностей открытых 
	$G_{\frac{1}{n}}$ и замкнутых $F_{\frac{1}{n}}$, которые и дают нужный 
	супремум или инфимум. Докажем последнее равенство.
	\begin{itemize}
		\item Пусть $A$ ограничено. Тогда 
\[
	\sup_{\substack{F \subset A \\ F \text{ замкнуто}}}{\lambda(F)} 
	= \sup_{\substack{K \subset A \\ K \text{ компакт}}}{\lambda(K)}
\]
	Левая часть равенства отличается от правой только тем, 
	что там разрешены неограниченные множества; но неограниченных подмножеств 
	ограниченного множества не бывает, поэтому получаем равенство.
		\item Пусть теперь $A$ неограничено.
			\begin{itemize}
				\item $\l(A) < +\infty$. Воспользуемся непрерывностью снизу:
\begin{align*}
	\Rm &= \bigcup{B(0, n)};~ A_n = A \cap B(0, n) \Lra A_1 \subset A_2 \subset \ldots\\
		&  \Lra \lim{\l(A_n)} = \l(A)
\end{align*}
				Начиная с некоторого места, $\l(A_n) > \l(A) - \e$, при этом, $A_n$ 
				ограничены. Выберем тот компакт (он есть по первому пункту), для которого \\
				$\l(K) > \l(A_n) - \e$. Тогда, поскольку $K \subset A_n \subset A$,
				имеем, что 
\[
		\l(A \setminus K) = \l((A \setminus A_n) \cup (A_n \setminus K)) < 2\e
\]
		\item $\l(A) = +\infty$. Доказывается аналогично, просто определение предела поменяется.
			Общий итог доказательства будет выглядеть так:
\[
	\forall R~ \exists A_n \colon~ \l(A_n) > R + 1~ \exists K \subset A_n\colon~ \l(K) > R
\]
			\end{itemize}
	\end{itemize}
\end{proof}

\begin{lemma}

    Пусть $\langle X', \cA', \mu' \rangle$ --- пространство с мерой.
    $\langle X, \cA, \_ \rangle$ --- заготовка для пространства с мерой.
    $T \colon X \to X'$ --- биекция, $\forall A \in \cA~ T(A) \in \cA'$,
    $T(\varnothing) = \varnothing$. Положим $\mu(A) = \mu'(T(A))$. Тогда
    $\mu$ --- мера на $\cA$.
\end{lemma}
\begin{proof}
	Докажем счетную аддитивность. Пусть $A = \bigsqcup{A_i}$. Тогда:
\[
	\mu(A) = \mu'(T(A)) = \mu'\left(T\left(\bigcup{A_i}\right)\right) 
	= \mu'\left(\bigcup{T(A_i)}\right) 
	= \sum{\mu'(T(A_i))} = \sum{\mu(A_i)}
\]
	Переход от образа объединения к объединению образов можно совершить из-за 
	того, что $T$ --- биекция.
\end{proof}

\begin{lemma}
	Пусть $A$ --- измеримое по Лебегу множество. Тогда $A$ представимо в виде 
\[
	A = \bigcup{K_i} \cup \mathfrak{N}
\]
	Где $K_i$ компакты и $\l(\mathfrak{N}) = 0$.
\end{lemma}
\begin{proof}
	Доказательство аналогично доказательству пункта (6) теоремы о свойствах меры Лебега 
	за исключением того, что пересечь или объединить компакты не получится.
\end{proof}

\begin{lemma}
    $T \colon \Rm \to \Rn \in C(\Rm)$, $\forall E \in \mathfrak{M}~ \lambda(E) = 0
    \Lra \lambda(T(E)) = 0$, \\ тогда $\forall A \in \mathfrak{M}~ T(A) \in \mathfrak{M}$.
\end{lemma}
\begin{proof}
	Пользуясь леммой, представим $A$ в виде:
\[
	A = \bigcup{K_i} \cup \mathfrak{N}
\]
	Тогда
\[
	T(A) = \bigcup{T(K_i)} \cup T(\mathfrak{N})
\]
	$T(K_i)$ --- непрерывный образ компакта, то есть компакт, а значит, борелевское множество,
	$T(\mathfrak{N})$ измеримо по условию, поэтому $T(A)$ измеримо как объединение измеримых 
	множеств.
\end{proof}

% preamble change fix
\let\x\undefined
\let\y\undefined
\let\h\undefined
\let\w\undefined
\let\z\undefined
\let\Od\undefined
\let\N\undefined
\let\C\undefined
% /preamble change fix
