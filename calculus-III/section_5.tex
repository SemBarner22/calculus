\section{Относительный экстремум}

\begin{definition}
    Пусть $f \colon E \subseteq \R^{m + n} \to \R$, $\phi \colon E \to \R^n$,
    $M_{\phi} = \{\,\x \in E \mid \phi(\x) = 0\,\}$, $\x_0 \in E$, $\phi(\x_0) =
    0$ называется точкой локального \textit{относительного} экстремума, если
    $x_0$ --- точка локального экстремума $f\big|_{M_{\phi}}$.
\end{definition}

\begin{theorem}(Необходимое условие относительного экстремума)

    Пусть $C^1 \ni f \colon E \subseteq \R^{m + n} \to \R$, $C^1 \ni \phi \colon
    E \to \R^n$, $\x_0 \in E$, $\phi(\x_0) = 0$, $\rank{\phi'(\x_0)} = n$,
    $\x_0$ --- точка локального экстремума, тогда $\exists \lambda \in \Rn
    \colon$
\[
    \begin{cases}
        f'(\x_0) - \lambda \cdot \phi'(\x_0) = 0 \\
        \phi(\x_0) = 0
    \end{cases}
\]
\end{theorem}
\begin{proof}
    TBD
\end{proof}

\begin{theorem}(Достаточное условие относительного экстремума)

    Пусть $C^1 \ni f \colon E \subseteq \R^{m + n} \to \R$, $C^1 \ni \phi \colon
    E \to \R^n$, $\x_0 \in E$, $\phi(\x_0) = 0$, $\rank{\phi'(\x_0)} = n$,
    выполнено необходимое условие относительного экстремума, то есть $\exists
    \lambda \in \Rn \colon$
 \[
     \begin{cases}
         f'(\x_0) - \lambda \cdot \phi'(\x_0) = 0 \\
         \phi(\x_0) = 0
     \end{cases}
 \]
    кроме того, пусть $\h = (\h_x \in \Rm, \h_y \in \Rn)$. Тогда, так как
    $\rank{\phi'(\x_0)}$, по $\h_x$ можно однозначно восстановить $\h_y$ такой,
    что $\phi'(\x_0)\h = 0$. Тогда рассмотрим квадратичную форму
\[
    Q(\h_x) = \d^2_{\x_0}{G(\h_x, \h_y)}
\]
    Где $G = f - \lambda \cdot \phi$ --- функция Лагранжа. В зависимости от
    определенности $Q$ можно сделать вывод о наличии экстремума в точке $x_0$:
    \begin{itemize}
        \item $Q$ положительно определена $\Lra \x_0$ --- точка относительного
        локального минимума.
        \item $Q$ отрицательно определена $\Lra \x_0$ --- точка относительного
        локального максимума.
        \item $Q$ неопределена $\Lra \x_0$ --- не точка экстремума.
        \item В остальных случаях требуется более детальное исследование.
    \end{itemize}
\end{theorem}
\begin{proof}
    TBD
\end{proof}

\begin{theorem}(Вычисление нормы линейного опреатора)

    Пусть $A \in Lin(\Rm, \Rn)$, $S$ --- множество собственных чисел $A^T A$.
    Тогда
\[
    \norm{A} = \max_{\lambda \in S}{\sqrt{\lambda}}
\]
\end{theorem}
