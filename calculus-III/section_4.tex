\section{Диффеоморфизмы}

\begin{definition}
    \textit{Областью} называют открытое связное множество.
\end{definition}

\begin{definition}
    Топологические пространства $X$, $Y$ \textit{гомеоморфны}, если
    существует обратимое и в обе стороны непрерывное $f \colon X \to Y$.
    $f$ называют \textit{гомеоморфизмом}.
\end{definition}

\begin{definition}
    \textit{Диффеоморфизмом} гладких многообразий $M, N$ называется
    обратимое и в обе стороны гладкое отображение $f \colon M \to N$.
\end{definition}

\begin{definition}
    Пусть $\Od$ --- область в $\Rm$. Тогда отображение $f \colon
    \Od \to \Rm$ называется \textit{диффеоморфизмом}, если оно обратимо
    и в обе стороны дифференцируемо.
\end{definition}

\begin{lemma}(О почти локальной инъективности)

    Пусть $f \colon \Od \subseteq \Rm \to \Rm$, $\x_0 \in \Od$, $f$
    дифференцируемо в $\x_0$, $\det{f'(\x_0)} \neq 0$, тогда \\ $\exists c, \delta
    > 0$ такие, что $\forall \h \colon \norm{\h} < \delta$~
$
    \norm{f(\x_0 + \h) - f(\x_0)} \geqslant c \norm{\h}
$
\end{lemma}
\begin{proof}
\[
    \norm{f(\x_0 + \h) - f(\x_0)} = \norm{f'(\x_0)\h + \a(\h)\norm{\h}}
    \geqslant \norm{f'(\x_0)\h} - \norm{\a(\h)\norm{\h}} \geqslant \frac{c}{2}
    \norm{\h}
\]
    Последнее неравенство выполнено по следующим причинам:
    \begin{itemize}
        \item $\norm{f'(\x_0)\h} \geqslant c \norm{\h}$, так как $f'(\x_0)$ обратим
        \item $\displaystyle \norm{\a(\h)\norm{\h}} \leqslant \frac{c}{2}
        \norm{\h}$ при достаточно малых $\h$, так как $\a(\h)$ --- бесконечно малое.
    \end{itemize}
\end{proof}

\begin{theorem}(О сохранении области)

    Пусть $f \colon \Od \subseteq \Rm \to \Rm$, $\forall \x \in \Od~ \det{f'(\x)}
    \neq 0$, тогда $f$ открыто.
\end{theorem}
\begin{proof}
    Достаточно рассмотреть случай, когда $\Od$ открыто. Тогда нужно показать, что
    $f(\Od)$ открыто. Зафиксируем $\x_0 \in \Od$ и $\y_0 = f(\x_0) \in f(\Od)$.
    По лемме о почти локальной инъективности имеем $c, \delta > 0$ такие, что
\[
    \forall \h \in \overline{B(0, \delta)}~ \norm{f(\x_0 + \h) - f(\x_0)}
    \geqslant c \norm{\h}
\]
    Положим
\[
    r = \frac{1}{2}\dist(\y_0, f(S(\x_0, \delta)))
\]
    Поскольку $f$ непрерывно, а сфера --- компакт, имеем, что $f(S(\x_0, \delta))$
    --- компакт. В свою очередь, $\r$ является метрикой, то есть непрерывно. Тогда
    $\r$ достигает минимума, то есть $r$ реализуется, а значит, не равно нулю
    (см. оценку выше). Раз $r > 0$, то $B(\y_0, r)$ --- полноправный шар,
    проверим, что он входит в образ $f$ целиком, что и закончит доказательство.
    Пусть $\y \in B(\y_0, r)$. Положим $g(\x) = \norm{f(\x) - \y}$ на
    $\overline{B(\x_0, \delta)}$; $g$ непрерывно, поэтому достигает минимума.
    Попробуем этот минимум найти:
\begin{itemize}
    \item Рассмотрим поведение $g(\x)$ на $S(\x_0, \delta)$:
\[
    g(\x) = \norm{f(\x) - \y} \geqslant \norm{f(\x) - \y_0} - \norm{\y_0 - \y}
    \geqslant 2r - r = r
\]
    \item $g(\x_0) = \norm{\y_0 - \y} < r$, так как $\y \in B(\y_0, r)$
\end{itemize}
    Тогда понятно, что минимум достигается не на границе. Раз так, он достигается
    во внутренности. Отображение $l\colon \x \mapsto g^2(\x)$ достигает минимума в
    той же точке, что и $g$, при этом $l'(\x) = f'(\x)(f(\x) - \y)$. Из
    невырожденности производного оператора следует, что $f(\x) = \y$, что и
    требовалось.
\end{proof}

\begin{figure}[ht]
    \centering
    \incfig{fopen}
    \caption{Теорема о сохранении области}
\end{figure}

\begin{corollary}

    Пусть $f \colon \Od \subseteq \Rm \to \R^{l \leqslant m}$, $\forall \x \in
    \Od~ \rank{f'(\x)} = l$, тогда $f$ открыто.
\end{corollary}
\begin{proof}
    Построим оторбражение $\tilde{f} \colon \Od \to \Rm$ следующим образом:
\begin{align*}
    &\tilde{f}_i = f_i,~ 1 \leqslant i \leqslant l \\
    &\tilde{f}_i = x_i, \text{ иначе}
\end{align*}
    Производный оператор тогда будет выглядеть так: \\
\[
    \tilde{f}' = \begin{pmatrix}
                    & & f' \\
                    0 & \cdots & 1 & \cdots & 0 \\
                    0 & \cdots & 0 & \cdots & 1 \\
                \end{pmatrix}
\]
    Где первая сверху единица стоит на $l + 1$-м месте. Тогда отображение
    $\tilde{f}$ тоже дифференцируемо, и его производный оператор невырожден.
    Применяя предыдущую теорему, получаем, что $\tilde{f}(A)$ открыто, если
    множество $A$ открыто. Тогда $f(A)$ тоже открыто.
\end{proof}

\begin{theorem}(О гладкости обратного отображения)

    Пусть $\Od \subseteq \Rm$ --- область, $T \in C^r(\Od, \Rm)$, $r \in \N \cup
    \{\,+\infty\,\}$, $\forall \x \in \Od~ \det{T'(\x)} \neq 0$, \\ $T$ обратимо,
    тогда $T^{-1} \in C^r$ и $(T^{-1})'(\y_0) = (T'(\x_0))^{-1}$, при $\y_0 =
    T(\x_0)$.
\end{theorem}
\begin{proof}
    Докажем теорему по индукции. В качестве базы рассмотрим случай $r = 1$.
    Обозначим $S = T^{-1}$, $S \colon T(\Od) \to \Rm$ непрерывно, так как
    по теореме о сохранении области $T$ открыто. Зафиксируем $\y_0 = T(\x_0)$
    и проверим дифференцируемость $S$ в точке $\y_0$.
    \begin{itemize}
        \item По теореме о почти локальной инъективности имеем
\[
        \exists c, \delta > 0\colon~ \forall \x \in B(\x_0, \delta)~
        \norm{T(\x) - T(\x_0)} \geqslant c \norm{\x - \x_0}
\]
        \item Воспользуемся дифференцируемостью $T$:
\[
        T(\x) - T(\x_0) = A(\x - \x_0) + \a(\x - \x_0) \norm{\x - \x_0}
\]
        Здесь $A = T'(\x_0)$. Положим $\y = T(\x)$:
\[
        \y - \y_0 = A(S(\y) - S(\y_0)) + \a(S(\y) - S(\y_0)) \norm{S(\y) - S(\y_0)}
\]
        Перепишем это равенство в виде, похожем на определение дифференцируемости
        $S$:
\[
        S(\y) - S(\y_0) = A^{-1}(\y - \y_0) + A^{-1} \a(S(\y) - S(\y_0))
        \norm{S(\y) - S(\y_0)}
\]
        Если мы поймем, что $\b(\y - \y_0) = A^{-1} \a(S(\y) - S(\y_0))
        \norm{S(\y) - S(\y_0)}$ --- бесконечно малое при $\y \to \y_0$,
        то мы получим определение дифференцируемости $S$ в точке $\y_0$.
        Проверим это:
\begin{align*}
    \b(\y - \y_0) &\leqslant \norm{A^{-1}} \norm{\a(S(\y) - S(\y_0))}
    \cdot \norm{\x - \x_0} \\
    &\leqslant \norm{A^{-1}} \norm{\a(S(\y) - S(\y_0))}
    \cdot \frac{1}{c} \norm{T(\x) - T(\x_0)} \\
    &= \norm{A^{-1}} \norm{\a(S(\y) - S(\y_0))}
    \cdot \frac{1}{c} \norm{\y - \y_0}
\end{align*}
        $S$ непрерывно, поэтому $\norm{\a(S(\y) - S(\y_0))}
        \xrightarrow[\y \to \y_0]{} 0$, тогда $\b(\y - \y_0)
        \xrightarrow[\y \to \y_0]{} 0$.
    \end{itemize}
    Теперь нужно доказать непрерывность $S'$. Из доказанного уже известно,
    что $S'(\y) = (T'(\x))^{-1}$:
\[
    \y \mapsto S(\y) = T^{-1}(\y) = \x \mapsto T'(\x) \mapsto (T'(\x))^{-1} = S'(\y)
\]
    Эту схему можно переписать в привычном виде:
\[
    S'(\y) = (T'(\x))^{-1} = (T'(S(\y)))^{-1}
\]
    Таким образом получаем, что $S'$ --- композиция непрерывных отображений, то
    есть непрерывно (в частности, отображение $GL(\Rm) \ni A \mapsto A^{-1}$
    непрерывно). Таким образом, база доказана.

    Для доказательства индукционного перехода нужно показать только
    гладкость $S$. Пусть $T \in C^n$, $S \in C^n$, покажем, что тогда если вдруг
    $T \in C^{n + 1}$, то и $S \in C^{n + 1}$. Для этого достаточно,
    чтобы $S' \in C^{n}$:
\[
    S'(\y) = (T'(\x))^{-1} = (T'(S(\y)))^{-1}
\]
    $S \in C^n$ по предположению индукции, $T' \in C^n$ потому, что
    $T \in C^{n + 1}$, обращение матрицы --- вообще класса $C^{\infty}$, то есть
    переход доказан.
\end{proof}

\begin{lemma}(О приближении оботражения его линеаризацией)

    Пусть $f \in C^1(\Od, \Rm)$, $\x_0 \in \Od$, тогда $\forall \h$
\[
    \norm{f(\x_0 + \h) - f(\x_0) - f'(\x_0)\h} \leqslant M \norm{\h}
\]
    где
\[
    M = \sup_{\z \in [\x_0, \x_0 + \h]}{\norm{f'(\z) - f'(\x_0)}}
\]
\end{lemma}
\begin{proof}
    Положим $F(\x) = f(\x) - f'(\x_0)(\x)$, тогда $F'(\x) = f'(\x) - f'(\x_0)$.
    Применим теорему Лагранжа к $F$:
\begin{align*}
    \norm{f(\x_0 + \h) - f(\x_0) - f'(\x_0)\h} &=
    \norm{F(\x_0 + \h) - F(\x_0)} \leqslant \sup_{\z \in [\x_0, \x_0 + \h]}
    {\norm{F'(\z)}} \cdot \norm{\h} \\
    &= \sup_{\z \in [\x_0, \x_0 + \h]} {\norm{f'(\z) - f'(\x_0)}} \cdot \norm{\h}
\end{align*}
\end{proof}

\begin{theorem}(О локальной обратимости)

    Пусть $f \in C^1(\Od, \Rm)$, $\x_0 \in \Od$, $\det{f'(\x_0)} \neq 0$,
    тогда $\exists U(\x_0)$ такая, что $f\big|_U$ --- диффеоморфизм.
\end{theorem}
\begin{proof}
    Если мы докажем, что $f$ обратимо в некоторой окрестности $U(\x_0)$,
    то по теореме о гладкости обратного отображения мы получим требуемое
    (невырожденность определителя в окрестности $\x_0$ следует из его непрерывности
    и того, что $\det{f'(\x_0)} \neq 0$). Для начала заметим. что из невырожденности
    оператора в точке $\x_0$ следует, что
\[
    \exists c > 0\colon~ \norm{f'(\x_0)\h} \geqslant c \norm{\h}
\]
    Попробуем построить окрестность. Пусть она будет содержать точки такие, что
    одновременно выполнены условия:
    \begin{itemize}
        \item $\displaystyle \norm{f'(\x) - f'(\x_0)} \leqslant \frac{c}{4}$.
            Эти точки есть вблизи $\x_0$ по теореме о непрерывно дифференцируемых
            отображениях.
        \item $\det{f'(\x)} \neq 0$.
            Такие точки есть из непрерывности $\det$.
    \end{itemize}
    Проверим, что в этой окрестности $f$ не склеивает точки, что и будет означать
    его обратимость; пусть $\y = \x + \h$, тогда:
\[
    f(\y) - f(\x) = (f(\y + \h) - f(\x) - f'(\x)\h) + (f'(\x) - f'(\x_0))\h
    + f'(\x_0)\h
\]
\begin{align*}
    \norm{f(\y) - f(\x)} &\geqslant \underbrace{\norm{f'(\x_0)} \norm{\h}}
    _{\geqslant c \norm{\h}} -
    \underbrace{\norm{f(\x + \h) - f(\x) - f'(\x)\h}}_{\leqslant M \norm{\h}
    \leqslant \frac{c}{2} \norm{h}}
    - \underbrace{\norm{f'(\x) - f'(\x_0)} \cdot \norm{\h}}_{\leqslant \frac{c}{4}
    \norm{\h}} \\
    &\geqslant \frac{c}{4} \norm{\h}
\end{align*}
    Где $M \norm{\h} \leqslant \frac{c}{2} \norm{\h}$ потому, что:
\[
    M \norm{\h} = \norm{\h} \sup{\norm{f'(\z) - f'(\x)}} \leqslant
    \sup{\big(\underbrace{\norm{f'(\z) - f'(\x_0)}}_{\leqslant \frac{c}{4}} + \underbrace{\norm{f(\x_0) - f(\x)}}_{\leqslant \frac{c}{4}}\big)} \leqslant
    \frac{c}{2}
\]
\end{proof}

\begin{theorem}(О неявном отображении)

    Пусть $\Od$ открыто, $f \colon \Od \subseteq \R^{m + n} \to \Rn$, $(\x \in
    \Rm, \y \in \Rn) \underset{f}{\mapsto} f(\x, \y)$, $f \in
    C^r$, \\ $(\ela, \elb) \in \Od \colon~ f(\ela, \elb) = \elemvec{0}$,
    $\det{f'_{\y}(\ela, \elb)} \neq 0$, тогда
    \begin{itemize}
        \item $\exists U(\ela)$, $\exists U(\elb)$, $\exists! \f \colon U(\ela)
        \to U(\elb) \in C^r$ такое, что $\forall \x \in U(\ela)~ f(\x, \f(\x)) =
        \elemvec{0}$ \item $\f'(\x) = -(f'_{\y}(\x, \f(\x)))^{-1} \cdot
        f'_{\x}(\x, \f(\x))$
    \end{itemize}
\end{theorem}
\begin{proof}
    \enewline
    \begin{itemize}
        \item Подготовим несколько объектов, полезных для доказательства.
        Положим
\begin{align*}
    &T \colon \Od \to \R^{m + n} \\
    &(\x, \y) \mapsto (\x, f(\x, \y))
\end{align*}
        Производный оператор этого отображения в блочном виде выглядит следующим
        образом:
\[
        T' = \begin{pmatrix}
                E & \elemvec{0} \\
                f'_x & f'_y
             \end{pmatrix}
\]
        Он обратим, так как $\det{T'} = 1 \cdot \det{f'_y} \neq 0$. Тогда по теореме о локальной
        обратимости $\exists U(\ela, \elb)\colon T\big|_U$
        --- диффеоморфизм. Без ограничения общности будем считать, что
        $U = \tilde{P} \times Q$, где $\tilde{P} \subset \Rm$, $Q \subset \Rn$
        открыты и $\ela \in \tilde{P}$, $\elb \in Q$. Будем обозначать
        $S = T^{-1}$ и $V = T(U)$ --- открыто в $\R^{m + n}$ так как $T$ --- диффеоморфизм.
        Заметим, что $T$ не меняет первую координату, то есть $S$ тоже её не меняет,
        а значит, имеет вид $S(\x, \y) = (\x, H(\x, \y))$, где $H(\x, \y) \in C^r$,
        так как $S \in C^r$.

        \item Пусть $P = (\Rm \times \elemvec{0}_n) \cap V$ --- открытое в
        $\Rm$ множество (само множество из $\R^{m + n}$, но нулевые координаты
        отбросим, см. рисунок). Предъявим требуемое отображение:
\[
    \f(\x) = H(\x, \elemvec{0})
\]

        \item Проверим, что $\f$ подходит: $f \in C^r$, так как
        $H \in C^r$. Проверим $f(\x, \f(\x)) = 0$ и единственность:
\[
    (\x, f(\x, \f(\x))) = T(\x, \f(\x)) = T(\x, H(\x, \elemvec{0})) =
    T(S(\x, \elemvec{0})) = (\x, \elemvec{0})
\]
        С другой стороны, если $f(\x, \y) = \elemvec{0}$, то:
\[
    (\x, \y) = S(T(\x, \y)) = S(\x, f(\x, \y)) = S(\x, \elemvec{0}) = H(\x, \elemvec{0})
\]
        То есть $\y = H(\x, \elemvec{0})$, из чего следует, что $\f$ единственно.

        \item Проверим второй пункт теоремы, вычислив производный оператор $\f$:
\[
    f(\x, \f(\x)) = \elemvec{0} \Lra \begin{pmatrix}
                                f'_x & f'_y
                           \end{pmatrix}
                           \cdot
                           \begin{pmatrix}
                               E \\
                               \f'
                           \end{pmatrix} = \elemvec{0}
\]
    Откуда получаем:
\[
    f'_x + f'_y \cdot \f' = \elemvec{0} \Lra \f' = -(f'_y)^{-1}f'_x
\]
    \end{itemize}
\end{proof}

\begin{figure}[ht]
    \centering
    \incfig{indirect}
    \caption{Теорема о неявном отображении}
\end{figure}

\begin{definition}
    $M \subseteq \Rm$ называют $k$\textit{-мерным многообразием} в $\Rm$,
    если оно локально гомеоморфно $\Rk$. Иными словами,
    $\forall \x \in M~ \exists U(\x)~ \exists \f \text{ --- гомеоморфизм}\colon
    \\ U(\x) \underset{\f}{\simeq} \Rk$.
\end{definition}

\begin{definition}
    $k$-мерное многообразие $M \subseteq \Rm$ называют \textit{простым}, если
    оно гомеоморфно $\Rk$. Иными словами, в предыдущем определении можно
    выбрать \\ $U(\x) = M$.
\end{definition}

\begin{definition}
    Пара $\langle U(\x), \f \rangle$ из определения называется \textit{картой},
    или \textit{параметризацией} многообразия в точке $\x$. Набор карт, который
    покрывает все $M$, называется \textit{атласом}.
\end{definition}

\begin{definition}
    Простое $k$-мерное многобразие $M$ называют $C^r$\textit{-гладким}, если \\
    $\f \in C^r$ --- параметризация $M$ и $\forall \x \in
    \Od~ \rank{\f'(\x)} = k$.
\end{definition}

\begin{theorem}(О задании гладкого многообразия системой уравнений)

    Пусть $M \subseteq \Rm$, $1 \leqslant k < m$, $r \in \N \cup
    \{\,+\infty\,\}$, тогда $\forall \elp \in M$ эквивалентны утверждения:
    \begin{itemize}
        \item $\exists U(\elp) \subseteq \Rm$ --- открытое такое, что
        $M \cap U$ --- простое $k$-мерное $C^r$-гладкое многообразие.
        \item $\exists \tilde{U}(\elp) \subseteq \Rm$ --- открытое такое,
        что $M \cap \tilde{U}$ можно задать системой $C^r$-гладких уравнений,
        иначе говоря: $\exists f_1, \ldots, f_{m - k} \colon \tilde{U} \to \R
        \in C^r$ такие, что \\ $\x \in M \cap \tilde{U} \Llra \forall i~
        f_i(\x) = 0$, причем $\{\,\grad{f_i}{\elp}\,\}$ линейно независим.
    \end{itemize}
\end{theorem}
\begin{proof}
    \enewline
    \item[$\Lra$] Пусть имеется параметризация $\f \colon \Od \subseteq \Rk \to \Rm
    \in C^r$, которая задает $M \cap U(\elp)$. Зафиксируем точку $t_0 = \f(\elp)$.
    Параметризация $C^r$ гладкая, поэтому $\rank{\f'} = k$. Будем считать,
    что ранг реализуется на первых $k$ строках. Рассмотрим проекцию
    $L \colon \x \mapsto (\x_1, \ldots, \x_k)$. Из того, что первые $k$ столбцов
    $\f'$ линейно независимы, имеем невырожденность производного оператора
    отображения $L \circ \f$. Тогда по теореме о локальной обратимости
    $\exists W(t_0) \colon (L \circ \f)\big|_W$ --- диффеоморфизм. Обозначим
    $V = (L \circ \f)(W)$. Тогда (см. рисунок) удобно интерпретировать $\f(W)$ как
    график отображения, заданного на $V$; $H \colon V \to \R^{m - k}$.
    Положим $\psi = (L \circ \f)^{-1}$, $\psi \in C^r$ по теореме о гладкости обратного
    отображения. Тогда вернемся к интерпретации графиком и посмотрим на точку
    $\x \in V$:
\[
    (\x, H(\x)) = \f(\psi(\x))
\]
    Тогда $H \in C^r$ как композиция $C^r$-гладких отображений. $\f$ гомеоморфизм,
    поэтому $\f(W)$ открыто в $M$. Тогда $\exists \tilde{U}$ открытое в $\Rm$
    такое, что $\f(W) = M \cap \tilde{U}$. Рассмотрим теперь набор функций:
\begin{align*}
    &f_i \colon \tilde{U} \to \R \\
    &\x \mapsto H_i(L(\x)) - x_{k + i}
\end{align*}
    для $i = 1\ldots m - k$. Тогда
\[
    \forall i = 1\ldots m - k~~ f_i(\x) = 0 \Llra
    f(\x) = \elemvec{0} \Llra H(L(\x)) - \x = \elemvec{0} \Llra
    H(L(\x)) = \x \Llra \x \in M \cap \tilde{U}
\]
    Осталось показать, что $\{\,\grad{f_i}{\elp}\,\}$ линейно независим.
    Для этого просто выпишем этот набор:
\begin{align*}
    \grad{f_i}{\elp} =
    \begin{pmatrix}
        \pderv{H_i}{\x_1} &  \cdots & \pderv{H_i}{\x_k} & 0 & \cdots & -1_{k + i} & \cdots & 0
    \end{pmatrix} \\
\end{align*}
    Очевидно, он линейно независим.
    \item[$\Lla$] Пусть теперь имеется система уравнений $f_i$. Составим из
    $f_i$ отображение $f \colon \Rm \to \R^{m - k}$. Поскольку набор градиентов
    линейно независим (будем считать, что ранг реализуется на последних $m - k$
    столбцах), можно применить теорему о неявном отображении:
    $\exists P(\elp_1, \ldots, \elp_k), Q(\elp_{k + 1}, \ldots, \elp_{m})$ и
    $\exists H \colon P \to Q$ такое, что
\[
    \forall (\x_1, \ldots, \x_k) \in P
    ~~ (\x_1, \ldots, \x_k, H(\x_1, \ldots, \x_k)) \text{ --- решение уравнения }
    f = \elemvec{0}
\]
    Построим теперь искомую параметризацию:
\begin{align*}
    &\f \colon P \to \Rm \\
    &\elu \mapsto (\elu, H(\elu))
\end{align*}
    Понятно, что $\f$ подходит в качестве параметризации $M \cap (P \times Q)$.
\end{proof}

\begin{figure}[ht]
    \centering
    \incfig{MU}
    \caption{Теорема о задании гладкого многообразия системой уравнений}
\end{figure}

\begin{corollary}(О двух параметризациях)

    Пусть $M$ --- $k$-мерное простое $C^r$-гладкое многообразие, $\elp \in M$,
    причем $C^r \ni \f_1 \colon \Od_1 \subseteq \Rk \to U \cap M$,
    $C^r \ni \f_2 \colon \Od_2 \subseteq \Rk \to U \cap M$ --- параметризации
    $U(\elp) \cap M$. Тогда $\f_1$ и $\f_2$ отличаются на диффеоморфизм, а
    именно, $\exists \psi \colon \Od_1 \to \Od_2$ --- диффеоморфизм, причем
    $\f_1 = \f_2 \circ \psi$.
\end{corollary}
\begin{proof}
    Будем считать, что невырожденность производных операторов параметризаций
    реализуется на первых $k$ строках. Пусть $L$ --- проекция из доказательства
    предыдущей теоремы. Тогда, как и в предыдущей теореме, отображения
    $L \circ \f_1$, $L \circ \f_2$, и им обратные существуют и гладкие.
    Тогда заметим, что $L$ обратимо (по крайней мере там, где обратимо $L \circ \f_1$).
    Положим тогда $\psi = (L \circ \f_2)^{-1} \circ (L \circ \f_1)$. В таком случае
    нетрудно видеть, что
\[
    \f_1 = \f_2 \circ (L \circ \f_2)^{-1} \circ (L \circ \f_1) =
    \f_2 \circ (\f_2^{-1} \circ L^{-1}) \circ (L \circ \f_1) = \f_1
\]
    $\psi$ обратим, потому что $\psi^{-1} = (L \circ \f_1)^{-1} \circ (L \circ \f_2)$.
    Гладкость $\psi$ и $\psi^{-1}$ следует из гладкости составных частей $\psi$.
\end{proof}

\begin{figure}[ht]
    \centering
    \incfig{two_param}
    \caption{Теорема о двух параметризациях}
\end{figure}

\newpage

\begin{definition}

    Пусть $M$ --- $C^r$-гладкое $k$-мерное многообразие в $\Rm$, $\elp \in M$,
    \\ $\f \colon \Od \subseteq \Rk \to \Rm$ --- параметризация окрестности
    $U(\elp)$, причем $\f(\ela) = \elp$. Тогда \textit{касательным
    пространством} к $M$ в точке $\elp$ называется $T_p(M) = \im{\f'(\ela)}$.
\end{definition}

\begin{theorem}(О корректности определения касательного пространства)

    Касательное пространство не зависит от выбора параметризации.
\end{theorem}
\begin{proof}
    Пусть $\f_1$ и $\f_2$ --- две параметризации $U(\elp)$. В таком случае по
    теореме о двух параметризациях $\exists C^r \ni \psi \colon \Od_1 \to \Od_2$
    такое, что $\f_1 = \f_2 \circ \psi$. В таком случае, поскольку
    $\psi$ --- диффеоморфизм, то есть имеет невырожденный производный оператор
    в $\ela$, имеем $\im{\psi'(\ela)} = \R^k$. Тогда:
\[
    \im{\f'_1(\ela)} = \im{\f'_2(\ela) \circ \psi'(\ela)} = \f'_2(\ela)(\R^k) = \im{\f'_2(\ela)}
\]
\end{proof}

\begin{theorem}(О касательном пространстве к гладкому пути)

    Пусть $M$ --- гладкое многообразие. Тогда $\v \in T_p(M) \Llra \exists
    \text{ гладкий путь } \gamma \colon [-1, 1] \to \Rm \colon~ \gamma([-1, 1])
    \subseteq M$ такой, что $\gamma(0) = \elp$ и $\gamma'(0) = \v$.
\end{theorem}
\begin{proof}
    \enewline
    \begin{itemize}
        \item Подготовим среду для доказательства. Зафиксируем $C^r \ni \f \colon
        \Od \subseteq \R^k \to M$ --- параметризация $M$ в какой-нибуль окрестности
        (не умаляя общности, во всем $M$). Вспомним в очередной раз отображение
        $L \colon \Rm \to \R^k$ --- проекцию первых $k$ координат. Копируя
        рассуждения предыдущих теорем, приходим к выводу, что $L \circ \f$ и
        $\psi = (L \circ \f)^{-1}$ лежат в классе $C^r$.
        \item[$\Lra$] Пусть имеется вектор $\v \in T_p(M)$, построим путь,
        проходящий через точку $\elp$ со скоростью $\v$. Пусть $\elp = \f(t_0)$,
        положим $u = (\f'(t_0))^{-1}(\v)$. По смыслу это та скорость, с которой
        нужно двигаться в $\Od$, чтобы получить скорость
        $\v$ в $\f(\Od)$ (то есть \textit{"прообраз скорости"}). Построим сам путь:
\[
    \gamma(s) = \f(t_0 + su)
\]
        Проверим его свойства:
\begin{align*}
    \gamma(0) &= \f(t_0) = \elp \\
    \gamma'(s)\big|_{s = 0} &= (\f'(t_0 + su) \cdot u)\big|_{s = 0}\\
    &= \f'(t_0) \cdot u = (\f'(t_0) \cdot (\f'(t_0)^{-1})(\v)) = \v
\end{align*}
        \item[$\Lla$] Пусть теперь имеется гладкий путь $\gamma \colon [-1, 1] \to
        M$ такой, что $\gamma(0) = \elp$ и $\gamma'(0) = \v$. Поймем, почему
        вектор $\v$ лежит в $T_p(M)$. Для этого воспользуемся подготовленным
        арсеналом и пустим путь \textit{"по кругу"}:
\[
    \gamma(s) = \f(\psi(L(\gamma(s))))
\]
        Проще всего понять это соотношение, внимательно посмотрев на картинку.
        Корректность же следует из свойств используемых отображений, в
        частности потому, что $L$ обратимо (на том же множестве, где обратимо
        $L \circ \f$). Теперь:
\[
    \gamma'(0) = \f'(\psi(L(\gamma(0))) \cdot (\ldots) =
    \f'(\psi(L(\elp))) \cdot (\ldots) = \f'(t_0) \cdot (\ldots) \in T_p(M)
\]
    \end{itemize}
\end{proof}

\begin{figure}[ht]
    \centering
    \incfig{mpath}
    \caption{Теорема о касательном пространстве к гладкому пути}
\end{figure}

\begin{theorem}(О касательном пространстве к графику функции)

    Касательное пространство к графику $C^r \ni f \colon \Od \subseteq \Rm \to
    \R$ в точке $\elp = (\x_0, f(\x_0))$ задается уравнением
\[
    y - f(\x_0) = f'_1(\x_1)(\x - \x_1) + \ldots + f'_m(\x_m)(\x - \x_m)
\]
\end{theorem}
\begin{proof}
    TBD
\end{proof}

\begin{theorem}(О касательном пространстве к поверхности уровня)

    Касательное пространство к поверхности уровня функции $f \colon \R^3 \to \R$
    задается уравнением
\[
    f'_x(x_0)(x - x_0) + f'_y(y_0)(y - y_0) + f'_z(z_0)(z - z_0) = 0
\]
\end{theorem}
\begin{proof}
    TBD
\end{proof}
