\section{Дифференцируемость и дифференциал отображений}

\begin{definition}
    Непустое множество $\O \subseteq \Rm$ называется \textit{областью},
    если оно открыто и связно.
\end{definition}

\begin{definition}
    Отображение $f \colon \O \to \Rn$, $\O$ --- область в $\Rm$ называется
    \textit{дифференцируемым} в точке $\x \in \O$, если существуют $\cA \in
    L(\Rm , \Rn)$, $r \colon \O \to \Rn$, такие что
\[
    f(\x + \h) = f(\x) + \cA \h + r(\h)
\]
    Где $r(\h)$ удовлетворяет уловию
\[
    \lim_{\h \to 0}{\frac{\norm{r(\h)}}{\norm{\h}}} = 0
\]
    или, что то же самое
\[
    f(\x + \h) = f(\x) + \cA \h + \a(\h)\norm{\h}
\]
    Где $\a(\h)$ бесконечно малое, то есть
\[
    \lim_{\h \to 0}{\a(\h)} = 0
\]
    или, что то же самое
\[
    \lim_{\h \to 0}{\frac{\norm{f(\x + \h) - f(\x) - \cA \h}}{\norm{\h}}} = 0
\]
\end{definition}

\begin{remark}
    Функции $\a$, $r$ из определения дифференцируемости зависят не только от $\h$,
    но и от $\x$.
\end{remark}

\begin{definition}
    Оператор $\cA \in L(\Rm, \Rn)$ из определения дифференцируемости будем
    называть \textit{дифференциалом} $f$ \textit{в точке} $\x$ и обозначать
    $\der{f}{\x} = \cA$.
\end{definition}

\begin{remark}
    Отображение $\x \mapsto \der{f}{\x}$, действующее из $\Rm$ в $L(\Rm, \Rn)$
    называют \textit{дифференциалом} $f$.
\end{remark}

\begin{definition}
    Матрицу, соответствующую производному оператору называют \textit{матрицей
    Якоби отображения} $f$ \textit{в точке} $\x$.
\end{definition}

\begin{theorem}(Единственность производной)

    $f \colon \O \to \Rn$, $\x \in \O$, $f$ дифференцируема в $\x$, тогда
    существует единственный производный оператор $f$ в точке $\x$.
\end{theorem}
\begin{proof}
    Проверим, что для любого $\z \in \Rm~ \cA \z$ задано однозначно.
     Пусть $\h = t\z$ при $t \in \R$:
\[
    f(\x + t \z) = f(\x) + \cA(t \z) + \a(t\z) \norm{t\z},~ t\z \to 0
\]
    Это эквивалентно
\[
    f(\x + t \z) = f(\x) + t\cA \z + t\a(t),~ t \to 0
\]
    Так как $\norm{\z}$ --- константа. Тогда
\begin{gather*}
    \cA\z = \frac{f(\x + t \z) - f(\x)}{t} - \a(t),~ t \to 0 \Llra \\
    \cA\z = \lim_{t \to 0}{\frac{f(\x + t \z) - f(\x)}{t}}
\end{gather*}
\end{proof}

\begin{proposition}(Производный оператор линейного отображения)

    Пусть $\cA \in L(\Rm, \Rn)$, тогда $\cA$ дифференцируемо в каждой точке и
    $\der{\cA}{\x} = \cA$
\end{proposition}
\begin{proof}
\[
    \cA(\x + \h) - \cA(\x) = \cA(\x + \h - \x) = \cA(\h)
\]
\end{proof}

\begin{proposition}(Линейность производного оператора)

    Пусть $f$, $g \colon \O \to \Rn$, $\O \subseteq \Rm$, дифференцируемы в
    $\x$. Тогда отображение $\a f + \b g$ дифференцируемо в точке $\x$, причем
    $\der{(\a f + \b g)}{\x} = \a \der{f}{\x} + \b \der{g}{\x}$
\end{proposition}
\begin{proof}
\begin{align*}
    &(\a f + \b g)(\x + \h) - (\a f + \b g)(\x) \\
    &= [(\a f)(\x + \h) - (\a f)(\x)] + [(\b g)(\x + \h) - (\b g)(\x)] \\
    &= [\der{f}{\x} + o] + [\der{g}{\x} + o] = \der{f}{\x} + \der{g}{\x} + o
\end{align*}
\end{proof}

\begin{theorem}(Дифференцируемость композиции)

    Пусть $f \colon \O \to \O_{1}$, $g \colon \O_{1} \to \Rk$, $\O \subseteq
    \Rm$, $\O_{1} \subseteq \Rn$ дифференцируемы в $\x$, тогда $F = g \circ f$
    дифференцируема в $\x$, причем $\der{F}{\x} = \der{g}{f(\x)} \cdot
    \der{f}{\x}$
\end{theorem}
\begin{proof}
    \begin{align*}
        g(f(\x + \h)) &= g(f(\x) + [\der{f}{\x}]\h + r(\h)) \\
        &= \{\, \v = [\der{f}{\x}]\h + r(\h) \,\} \\
        &= g(f(\x) + \v) = g(f(\x)) + [\der{g}{f(\x)}]\v + \widetilde{r}(\v) \\
        &= g(f(\x)) + [\der{g}{f(\x)}][\der{f}{\x}]\h + [\der{g}{f(\x)}]r(\h) +
        \widetilde{r}(\v) \\
    \end{align*}
    Осталось показать, что $[\der{g}{f(\x)}]r(\h) + \widetilde{r}(\v) =
    o(\norm{\h})$.
\[
    \norm{[\der{g}{f(\x)}]r(\h) + \widetilde{r}(\v)} \leqslant
    \norm{[\der{g}{f(\x)}]\frac{r(\h)}{\norm{r(\h)}}\norm{r(\h)}} +
    \widetilde{\a}(\norm{\v})\norm{\v}
\]
    Обозначим $\w = \frac{r(\h)}{\norm{r(\h)}}$, причем $\norm{\w} = 1$.
    Из определения нормы оператора получаем
\[
    \norm{[\der{g}{f(\x)}]\w} \leqslant \norm{\der{g}{f(\x)}}
\]
    Кроме того, $\norm{\v} \leqslant \norm{\der{f}{\x}}\h +
    \a(\norm{\h})\norm{\h}$. Окончательно получаем
\[
    \norm{[\der{g}{f(\x)}]r(\h) + \widetilde{r}(\v)} \leqslant
    \norm{\der{g}{f(\x)}}\norm{r(\h)} + \widetilde{\a}(\norm{\der{f}{\x}} \cdot
    \norm{\h} + \a(\norm{\h}) \cdot \norm{\h})\norm{\v} \leqslant
    \b(\norm{\h})\norm{\h}
\]
    Для некоторой $\b(t) \xrightarrow[t \to 0]{} 0$.
\end{proof}

\begin{definition}
    Пусть $\O \subseteq \Rm$, $f \colon \O \to \Rn$, $\{\, \elemvec{u}_1, \ldots,
    \elemvec{u}_n \,\}$ --- стандартный базис $\Rn$, тогда отображения
\[
    f_i(x) \defeq \langle f(x), \elemvec{u}_i \rangle
\]
    где $f_i \colon \O \to \R$, называются \textit{координатными функциями}.
\end{definition}

\begin{theorem}(Дифференцируемость координатных функций)

    Пусть $\O \subseteq \Rm$, $f \colon \O \to \Rn$, $\x \in \O$, тогда
\[
    f \text{ дифференцируемо в } \x \Llra \forall i~ f_i \text{ дифференцируемо в
    } \x
\]
    причем
\[
    \der{f}{\x} = \begin{pmatrix}
                           \der{f_1}{\x} \\
                           \vdots \\
                           \der{f_n}{\x}
                  \end{pmatrix}
\]
\end{theorem}
\begin{proof}
    \enewline
    \begin{itemize}
        \item[$\Lla$]
\[
    f(x) = \sum_{i=1}^{n}{f_i(\x)\elemvec{u}_i} = \sum_{i=1}^{n}{g_i(f_i(\x))}
\]
        где $g_i(t) = t \elemvec{u}_i$ --- линейно, то есть дифференцируемо.
        Тогда $f$ дифференцируемо как сумма композиций дифференцируемых функций.
        \item[$\Lra$] $f_i$ дифференцируемы как композиции $f$ и соответствующей
        проекции (проекция линейна, то есть дифференцируема).
    \end{itemize}
\end{proof}

\begin{definition}
    Пусть $f \colon \O \to \Rn$, $\O \in \Rm$ --- область, тогда
    \textit{производной по направлению} $\elemvec{u} \in \Rn$ в точке $\x$
    называется
\[
    \dder{f}{\x}{\elemvec{u}} \defeq \lim_{\substack{t \to 0 \\ \x + t\elemvec{u}
    \in \O}}{\frac{f(\x + t\elemvec{u}) - f(\x)}{t}}
\]
    если он существует.
\end{definition}

\begin{definition}
    Пусть $f \colon \O \to \Rn$, $\O \in \Rm$, $\{\, \elemvec{u}_1, \ldots,
    \elemvec{u}_n \,\}$ --- стандартный базис $\Rm$, тогда \textit{частной
    производной $f$ по $k$ -й переменной} называется
\[
    \pderv{f}{x_k}(\x) \defeq f'_k \defeq \pderi{f}{k}(\x) \defeq
    \dder{f}{\x}{\elemvec{u}_k}
\]
\end{definition}

\begin{proposition}
    Пусть $f \colon \O \subseteq \Rm \to \Rn$, $\x \in \O$, $\elemvec{u} \in
    \Rm$. Тогда
\[
    \exists \dder{f}{\x}{\elemvec{u}} = [\der{f}{\x}]\elemvec{u}
\]
\end{proposition}
\begin{proof} Для любых $t \in \R$, $\x + t\elemvec{u} \in \O$ имеем
\begin{align*}
    &\frac{f(\x + t\elemvec{u}) - f(\x)}{t} =
    \frac{[\der{f}{\x}](t\elemvec{u}) + r(t\elemvec{u})}{t} \\
    &= [\der{f}{\x}]\elemvec{u} +
    \frac{r(t\elemvec{u})}{t} \leqslant [\der{f}{\x}]\elemvec{u} +
    \frac{\a(\elemvec{u})\norm{t\elemvec{u}}}{t} =
    [\der{f}{\x}]\elemvec{u} + \a(\elemvec{u})\norm{\elemvec{u}}
\end{align*}
\end{proof}

\begin{theorem}(Вид матрицы Якоби)
    Пусть $f \colon \O \subseteq \Rm \to \Rn$, $\x \in \O$, тогда
\[
    \der{f}{\x} = \begin{pmatrix}
                    \dder{f_1}{\x}{1} & \dder{f_1}{\x}{2} & \cdots &
                    \dder{f_1}{\x}{m} \\
                    \dder{f_2}{\x}{1} & \dder{f_2}{\x}{2} & \cdots &
                    \dder{f_2}{\x}{m} \\
                    \vdots & \vdots & \cdots & \vdots \\
                    \dder{f_n}{\x}{1} & \dder{f_n}{\x}{2} & \cdots &
                    \dder{f_n}{\x}{m}
                  \end{pmatrix}
\]
\end{theorem}
\begin{proof}
    Пусть $(\elemvec{e}_i)$ --- базис $\Rm$, $(\tilde{\elemvec{e}}_i)$ --- базис
    $\Rn$, тогда
\begin{align*}
    [\der{f}{\x}]_{i, j} = \langle [\der{f}{\x}]\elemvec{e}_i,
    \tilde{\elemvec{e}}_j \rangle
    &= \left\langle \frac{[\der{f}{\x}](t\elemvec{e}_i)}{t},
    \tilde{\elemvec{e}}_j \right\rangle
    = \left\langle \lim_{t \to 0}{\frac{f(\x + t\elemvec{e}_i) - f(\x)}{t}},
    \tilde{\elemvec{e}}_j \right\rangle \\
    &= \lim_{t \to 0}{\left\langle \frac{f(\x + t\elemvec{e}_i) - f(\x)}{t},
    \tilde{\elemvec{e}}_j \right\rangle}
    = \lim_{t \to 0}{\frac{f_j(\x + t\elemvec{e}_i) - f_j(\x)}{t}}
    = \dder{f_j}{\x}{i}
\end{align*}
\end{proof}

\begin{proposition}
    Пусть $f \colon \Rm \to \R$, $\elemvec{u} \in \Rm$. Тогда
\[
    \dder{f}{\x}{\elemvec{u}} = \langle \grad{f}{\x}, \elemvec{u} \rangle
\]
\end{proposition}
\begin{proof}
    \enewline
\[
    \dder{f}{\x}{\elemvec{u}} = [\der{f}{\x}]\elemvec{u} =
    [\grad{f}{\x}]\elemvec{u} = \langle \grad{f}{\x}, \elemvec{u} \rangle
\]
\end{proof}

\begin{proposition}(Необходимое условие дифференцируемости)

    Если $f$ дифференцируемо в $\x$, то существуют все частные производные в
    точке $\x$, причем матрица Якоби $f$ в точке $\x$ совпадает с матрицей,
    составленной из матриц якоби $f_i$ в точке $\x$:
\[
    \der{f}{\x} = \left(\pderv{f}{\x_1}(\x), \ldots, \pderv{f}{\x_m}(\x)\right)
\]
\end{proposition}
\begin{proof}

    Подставим в определение дифференцируемости $\h = t \elemvec{e}_k$:
\begin{align*}
    f(\x + t \elemvec{e}_k) &= f(\x) + [\der{f}{\x}](t \elemvec{e}_k) +
    \a(\h)\norm{\h} \\ &= f(\x) + t[\der{f}{\x}]\elemvec{e}_k +
    \a(\h)\norm{\h}
\end{align*}
    Отсюда по определению частной производной получаем
\[
    \pderv{f}{x_k}(\x) \defeq \lim_{t \to 0}{\frac{f(\x +
    t\elemvec{e}_k) - f(\x)}{t}} = [\der{f}{\x}]\elemvec{e}_k
\]
\end{proof}

\begin{theorem}(Достаточное условие дифференцируемости)

    Пусть $f \colon \O \subseteq \Rm \to \R$, $\ela \in \O$, $B(\ela) \subseteq
    \O$, В $B(\ela)$ существуют все частные производные, причем они непрерывны в
    точке $\ela$. Тогда $f$ дифференцируемо в точке $\ela$.
\end{theorem}
\begin{proof}
    Докажем теорему для случая $m = 2$. Схема, примененная в доказательстве
    тривиально обобщается на произвольные $m$.

\begin{align*}
    f(\x_1, \x_2) - f(\ela_1, \ela_2) &= (f(\x_1, \x_2) - f(\ela_1, \x_2))
    + (f(\ela_1, \x_2) - f(\ela_1, \ela_2)) \\
    &\overset{\text{Лагранж}}{=} f'_{\x_1}(\tilde{\x}_1, \x_2)(\x_1 - \ela_1) +
    f'_{\x_2}(\x_1, \tilde{\x}_2)(\x_2 - \ela_2) \\
    &= f'_{\x_1}(\ela_1, \ela_2)(\x_1 - \ela_1) + f'_{\x_2}(\ela_1, \ela_2)(\x_2
    - \ela_2) \\
    &+ \left[\underbrace{(f'_{\x_1}(\tilde{\x}_1, \x_2) - f'_{\x_1}(\ela_1,
    \ela_2))}_{\to 0 \text{ по непрерывности}}\frac{\x_1 - \ela_1}{\norm{\x -
    \ela}} + \underbrace{(f'_{\x_2}(\x_1, \tilde{\x}_2) - f'_{\x_2}(\ela_1,
    \ela_2))}_{\to 0 \text{ по непрерывности}}\frac{\x_2 - \ela_2}{\norm{\x -
    \ela}}\right]\norm{\x - \ela} \\
    &= f'_{\x_1}(\ela_1, \ela_2)(\x_1 - \ela_1) + f'_{\x_2}(\ela_1, \ela_2)(\x_2
    - \ela_2) + \a(\x - \ela)\norm{\x - \ela}
\end{align*}
\end{proof}

\begin{theorem}

    Пусть $f$,$g \colon \O \subseteq \Rm \to \Rn$, $\lambda \colon \O \to
    \R$, $\ela \in \O$, $f$,$g$,$\lambda$ дифференцируемы в $\ela$. Тогда
    $\lambda f$, $\langle f, g \rangle$ дифференцируемы в $\ela$, причем
    \begin{itemize}
        \item $[\der{(\lambda f)}{\ela}]\h =
        [\der{\lambda}{\ela}]\h \cdot f(\ela) + \lambda(\ela) \cdot
        [\der{f}{\ela}]\h$
        \item $[\der{\langle f, g \rangle}{\ela}]\h =
        \langle [\der{f}{\ela}]\h, g(\ela) \rangle
        + \langle f(\ela), [\der{g}{\ela}]\h \rangle$

    \end{itemize}
\end{theorem}
\begin{proof}
    \enewline
    \begin{itemize}
        \item Докажем покоординатно:
\begin{align*}
    & (\lambda f_i)(\ela + \h) - (\lambda f_i)(\ela) \\
    &= \lambda(\ela + \h) \cdot f_i(\ela + \h) - (\lambda f_i)(\ela) \\
    &= (\lambda(\ela) + \der{\lambda}{\ela}\h + \a(\h)\norm{\h}) \cdot
    (f_i(\ela) + \der{f_i}{\ela}\h + \b(\h)\norm{\h}) - (\lambda f_i)(\ela) \\
    &= [\der{\lambda}{\ela}] \cdot f_i(\ela) + \lambda(\ela) \cdot
    [\der{f_i}{\ela}]\h + o(\h)
\end{align*}
        \item
\begin{align*}
    & [\der{\langle f, g \rangle}{\ela}]\h = \der{\left[\sum_{i =
    1}^n{f_i \cdot g_i}\right]}{\ela}\h = \sum_{i =
    1}^n{\der{\left[{f_i \cdot g_i}\right]}{\ela}}\h =
    \sum_{i = 1}^{n}{\left([\der{f_i}{\ela}]\h \cdot g_i(\ela) + f_i(\ela) \cdot
    [\der{g_i}{\ela}]\h\right)} \\
    &= \sum_{i = 1}^{n}{\left([\der{f_i}{\ela}]\h \cdot g_i(\ela)\right)} +
    \sum_{i = 1}^{n}{\left(f_i(\ela) \cdot [\der{g_i}{\ela}]\h\right)} =
    \langle [\der{f}{\ela}]\h, g(\ela) \rangle + \langle f(\ela),
    [\der{g_i}{\ela}]\h \rangle
\end{align*}
    \end{itemize}
\end{proof}

\begin{definition}
    Пусть $f \colon \O \subseteq \Rm \to \R$, $\x \in \O$. Тогда
    \textit{градиентом} $\f$ в точке $\x$ наывается вектор
\[
    \grad{f}{\x} \defeq \begin{pmatrix}
        \pderv{f}{x_1}(\x) \\
        \vdots \\
        \pderv{f}{x_m}(\x)
    \end{pmatrix}
\]
\end{definition}

\begin{theorem}(Экстремальное свойство градиента)

    Пусть $f \colon \O \subseteq \Rm \to \R$, $\x \in \O$, $f$ дифференцируемо
    в $\x$, $\grad{f}{\x} \neq 0$ Тогда
\[
    \elemvec{l} = \frac{\grad{f}{\x}}{\norm{\grad{f}{\x}}}
\]
    --- направление наибольшего возрастания $f$, то есть
\[
    \forall \h \in \Rm, \norm{\h} = 1 \Lra \dder{f}{\x}{\h} \leqslant
    \dder{f}{\x}{\elemvec{l}}
\]
\end{theorem}
\begin{proof}
    \enewline
\[
    \dder{f}{\x}{\h} = [\der{f}{\x}]\h = [\grad{f}{\x}]\h =
    \langle \grad{f}{\x}, \h \rangle \leqslant \norm{\grad{f}{\x}} \norm{\h}
    = \norm{\grad{f}{\x}}
\]
\end{proof}

\newpage
\section{Теоремы Лагранжа для отображений}

\begin{theorem}(Лагранжа для векторнозначных функций)

    Пусть $f \in C([a, b], \Rn)$, дифференцируемо на $(a, b)$. Тогда
\[
    \exists c \in (a,b) \colon~ \norm{f(b) - f(a)} \leqslant \norm{f'(c)} |b - a|
\]
\end{theorem}
\begin{proof}
    Положим $\f(x) = \langle f(b) - f(a), f(t) - f(a) \rangle$. Тогда
\[
    \f(a) = 0, \f(b) = \langle f(b) - f(a), f(b) - f(a) \rangle = \norm{f(b) -
    f(a)}^2
\]
    Применим теорему Лагранжа для $\f$:
\begin{align*}
    \exists c \in (a, b) \colon~ &\norm{f(b) - f(a)}^2 = \f(b) - \f(a)
    \underset{\text{Лагранж}}{=} \f'(c) |b - a| =
    \langle f(b) - f(a), f'(c) \rangle \cdot |b - a| \\
    &\underset{\text{КБШ}}{\leqslant} \norm{f(b) - f(a)} \cdot \norm{f'(c)}
    \Lra \norm{f(b) - f(a)} \leqslant \norm{f'(c)} |b - a|
\end{align*}
\end{proof}

\begin{theorem}(Лагранжа для отображений)

    Пусть $f \colon \O \subseteq \Rm \to \Rn$, дифференцируемо на $\O$,
    $[\ela, \elb] \subseteq \O$, тогда
\[
    \norm{f(\elb) - f(\ela)} \leqslant \sup_{\x \in [\ela,
    \elb]}{\norm{\der{f}{\x}}} \cdot \norm{\elb - \ela}
\]
\end{theorem}
\begin{proof}

    $g = f(a + t(b - a))$ для $t \in [0, 1]$ --- дифференцируемо как композиция
    дифференцируемых функций. По предыдущей теореме
\begin{align*}
    \exists t_0 \in (0, 1)\colon~ \norm{g(1) - g(0)} &\leqslant \norm{g'(t_0)}
    = \norm{f'(\ela + t_0(\elb - \ela)) \cdot (\elb - \ela)} \\
    &\leqslant \norm{f'(\ela + t(\elb - \ela))}
    \norm{\elb - \ela} \leqslant \sup_{\x \in [\ela,
    \elb]}{\norm{\der{f}{\x}}} \cdot \norm{\elb - \ela}
\end{align*}
\end{proof}
