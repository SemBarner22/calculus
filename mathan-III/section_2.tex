\section{Дифференцируемость и дифференциал функций нескольких переменных}

\begin{definition}
    Непустое множество $\O \subseteq \Rm$ называется \textit{областью},
    если оно открыто и связно.
\end{definition}

\begin{definition}
    Отображение $f \colon \O \to \Rn$, $\O$ --- область в $\Rm$ называется
    \textit{дифференцируемым} в точке $\x \in \O$, если существуют $\cA \in
    L(\Rm , \Rn)$, $\a \colon \O \to \Rn$ --- бесконечно малое в точке $\x$,
    такие что
\[
    f(\x + \h) = f(\x) + \cA \h + \a(\h) \norm{\h},~\h \to 0
\]
    или, что то же самое
\[
    \lim_{\h \to 0}{\frac{\norm{f(\x + \h) - f(\x) - \cA \h}}{\norm{\h}}} = 0
\]
\end{definition}

\begin{remark}
    Функция $\a$ из определения дифференцируемости зависит не только от $\h$,
    но и от $\x$.
\end{remark}

\begin{definition}
    Оператор $\cA \in L(\Rm, \Rn)$ из определения дифференцируемости будем
    называть \textit{дифференциалом} $f$ \textit{в точке} $\x$ и обозначать
    $\der{f}{\x} = \cA$.
\end{definition}

\begin{remark}
    Отображение $\x \mapsto \der{f}{\x}$, действующее из $\Rm$ в $L(\Rm, \Rn)$
    называют \textit{дифференциалом} $f$.
\end{remark}

\begin{definition}
    Матрицу, соответствующую производному оператору называют \textit{матрицей
    Якоби отображения} $f$ \textit{в точке} $\x$.
\end{definition}

\begin{theorem}(Единственность производной)

    $f \colon \O \to \Rn$, $\x \in \O$, $f$ дифференцируема в $\x$, тогда
    существует единственный производный оператор $f$ в точке $\x$.
\end{theorem}
\begin{proof}
    Проверим, что для любого $\z \in \Rm~ \cA \z$ задано однозначно.
     Пусть $\h = t\z$ при $t \in \R$:
\[
    f(\x + t \z) = f(\x) + \cA(t \z) + \a(\norm{t\z}) \norm{t\z},~ t\z \to 0
\]
    Это эквивалентно
\[
    f(\x + t \z) = f(\x) + t\cA \z + t\a(t),~ t \to 0
\]
    Так как $\norm{\z}$ --- константа. Тогда
\begin{gather*}
    \cA\z = \frac{f(\x + t \z) - f(\x)}{t} - \a(t),~ t \to 0 \Llra \\
    \cA\z = \lim_{t \to 0}{\frac{f(\x + t \z) - f(\x)}{t}}
\end{gather*}
\end{proof}

\begin{proposition}(Производный оператор линейного отображения)

    Пусть $\cA \in L(\Rm, \Rn)$, тогда $\cA$ дифференцируемо в каждой точке и
    $\der{\cA}{\x} = \cA$
\end{proposition}
\begin{proof}
\[
    \cA(\x + \h) - \cA(\x) = \cA(\x + \h - \x) = \cA(\h)
\]
\end{proof}

\begin{proposition}(Линейность производного оператора)

    Пусть $f\!$, $\!\!g \colon \O \to \Rn$, $\O \subseteq \Rm$, дифференцируемы в
    $\x$. Тогда отображение $\a f + \b g$ дифференцируемо в точке $\x$, причем
    $\der{(\a f + \b g)}{\x} = \a \der{f}{\x} + \b \der{g}{\x}$
\end{proposition}
\begin{proof}
\begin{align*}
    &(\a f + \b g)(\x + \h) - (\a f + \b g)(\x) \\
    &= [(\a f)(\x + \h) - (\a f)(\x)] + [(\b g)(\x + \h) - (\b g)(\x)] \\
    &= [\der{f}{\x} + o] + [\der{g}{\x} + o] = \der{f}{\x} + \der{g}{\x} + o
\end{align*}
\end{proof}

\begin{theorem}(Дифференцируемость композиции)

    Пусть $f \colon \O \to \O_{1}$, $g \colon \O_{1} \to \Rk$, $\O \subseteq
    \Rm$, $\O_{1} \subseteq \Rn$ дифференцируемы в $\x$, тогда $F = g \circ f$
    дифференцируема в $\x$, причем $\der{F}{\x} = \der{g}{f(\x)} \cdot
    \der{f}{\x}$
\end{theorem}
\begin{proof}
    \begin{align*}
        g(f(\x + \h)) &= g(f(\x) + [\der{f}{\x}]\h + r(\h)) \\
        &= \{\, \v = [\der{f}{\x}]\h + r(\h) \,\} \\
        &= g(f(\x) + \v) = g(f(\x)) + [\der{g}{f(\x)}]\v + \widetilde{r}(\v) \\
        &= g(f(\x)) + [\der{g}{f(\x)}][\der{f}{\x}]\h + [\der{g}{f(\x)}]r(\h) +
        \widetilde{r}(\v) \\
    \end{align*}
    Осталось показать, что $[\der{g}{f(\x)}]r(\h) + \widetilde{r}(\v) =
    o(\norm{\h})$.
\[
    \norm{[\der{g}{f(\x)}]r(\h) + \widetilde{r}(\v)} \leqslant
    \norm{[\der{g}{f(\x)}]\frac{r(\h)}{\norm{r(\h)}}\norm{r(\h)}} +
    \widetilde{\a}(\norm{\v})\norm{\v}
\]
    Обозначим $\w = \frac{r(\h)}{\norm{r(\h)}}$, причем $\norm{\w} = 1$.
    Из определения нормы оператора получаем
\[
    \norm{[\der{g}{f(\x)}]\w} \leqslant \norm{\der{g}{f(\x)}}
\]
    Кроме того, $\norm{\v} \leqslant \norm{\der{f}{\x}}\h +
    \a(\norm{\h})\norm{\h}$. Окончательно получаем
\[
    \norm{[\der{g}{f(\x)}]r(\h) + \widetilde{r}(\v)} \leqslant
    \norm{\der{g}{f(\x)}}\norm{r(\h)} + \widetilde{\a}(\norm{\der{f}{\x}}\h +
    \a(\norm{\h})\norm{\h})\norm{\v} \leqslant \b(\norm{\h})\norm{\h}
\]
    Для некоторой $\b(t) \xrightarrow[t \to 0]{} 0$.
\end{proof}

\begin{definition}
    Пусть $\O \subseteq \Rm$, $f \colon \O \to \Rn$, $\{\, \elemvec{u}_1, \ldots,
    \elemvec{u}_n \,\}$ --- стандартный базис $\Rn$, тогда отображения
\[
    f_i(x) \defeq \langle f(x), \elemvec{u}_i \rangle
\]
    где $f_i \colon \O \to \R$, называются \textit{координатными функциями}.
\end{definition}

\begin{theorem}(Дифференцируемость координатных функций)

    Пусть $\O \subseteq \Rm$, $f \colon \O \to \Rn$, $\x \in \O$, тогда
\[
    f \text{ дифференцируемо в } \x \Llra \forall i~ f_i \text{ дифференцируемо в
    } \x
\]
\end{theorem}
\begin{proof}
    \enewline
    \begin{itemize}
        \item[$\Lla$]
\[
    f(x) = \sum_{i=1}^{n}{f_i(\x)\elemvec{u}_i} = \sum_{i=1}^{n}{g_i(f_i(\x))}
\]
        где $g_i(t) = t \elemvec{u}_i$ --- линейно, то есть дифференцируемо.
        Тогда $f$ дифференцируемо как сумма композиций дифференцируемых функций.
        \item[$\Lra$] $f_i$ дифференцируемы как композиции $f$ и соответствующей
        проекции (проекция линейна, то есть дифференцируема).
    \end{itemize}
\end{proof}

\begin{definition}
    Пусть $f \colon \O \to \Rm$, $\O \in \Rn$ --- область, тогда
    \textit{производной по направлению} $\elemvec{u} \in \Rn$ в точке $\x$ называется
\[
    \dder{f}{\x}{\elemvec{u}} \defeq \lim_{\substack{t \to 0 \\ \x + t\elemvec{u}
    \in \O}}{\frac{f(\x + t\elemvec{u}) - f(\x)}{t}}
\]
    если он существует.
\end{definition}

\begin{definition}
    Пусть $f \colon \O \to \Rm$, $\O \in \Rn$, $\{\, \elemvec{u}_1, \ldots,
    \elemvec{u}_n \,\}$ --- стандартный базис $\Rn$, тогда \textit{частной
    производной $f$ по $k$ -й переменной} называется
\[
    \pderv{f}{x_k}(\x) \defeq f'_k \defeq \pderi{f}{k}(\x) \defeq
    \dder{f}{\x}{\elemvec{u}_k}
\]
\end{definition}

\begin{proposition}(Необходимое условие дифференцируемости)

    Если $f$ дифференцируема в $\x$, то все $f_i$ дифференцируемы в $\x$, причем
    матрица Якоби $f$ в точке $\x$ совпадает с матрицей, составленной из матриц
    якоби $f_i$ в точке $\x$:
\[
    \der{f}{\x} = (\der{f_1}{\x}, \ldots, \der{f_n}{\x})
\]
\end{proposition}
\begin{proof}

    Подставим в определение дифференцируемости $\h = t \elemvec{e}_k$:
\begin{align*}
    f(\x + t \elemvec{e}_k) &= f(\x) + [\der{f}{\x}](t \elemvec{e}_k) +
    \a(\norm{\h})\norm{\h} \\ &= f(\x) + t[\der{f}{\x}]\elemvec{e}_k +
    \a{\norm{\h}}\norm{\h}
\end{align*}
    Отсюда по определению частной производной получаем
\[
    \pderv{f}{x_k}(\x) \defeq \lim_{t \to 0}{\frac{f(\x +
    t\elemvec{e}_k) - f(\x)}{t}} = [\der{f}{\x}]\elemvec{e}_k
\]
\end{proof}

\begin{theorem}(Достаточное условие дифференцируемости)

    Пусть $f \colon \O \subseteq \Rm \to \R$, $\ela \in \O$, $B(\ela) \subseteq
    \O$, В $B(\ela)$ существуют все частные производные, причем они непрерывны в
    точке $\ela$. Тогда $f$ дифференцируема в точке $\ela$.
\end{theorem}
\begin{proof}
    Докажем теорему для случая $m = 2$. Схема, примененная в доказательстве
    тривиально обобщается на произвольные $m$.

\begin{align*}
    f(\x_1, \x_2) - f(\ela_1, \ela_2) &= (f(\x_1, \x_2) - f(\ela_1, \x_2))
    + (f(\ela_1, \x_2) - f(\ela_1, \ela_2)) \\
    &\overset{\text{Лагранж}}{=} f'_{\x_1}(\tilde{\x}_1, \x_2)(\x_1 - \ela_1) +
    f'_{\x_2}(\x_1, \tilde{\x}_2)(\x_2 - \ela_2) \\
    &= f'_{\x_1}(\ela_1, \ela_2)(\x_1 - \ela_1) + f'_{\x_2}(\ela_1, \ela_2)(\x_2
    - \ela_2) \\
    &+ \left[\underbrace{(f'_{\x_1}(\tilde{\x}_1, \x_2) - f'_{\x_1}(\ela_1,
    \ela_2))}_{\to 0 \text{ по непрерывности}}\frac{\x_1 - \ela_1}{\norm{\x -
    \ela}} + \underbrace{(f'_{\x_2}(\x_1, \tilde{\x}_2) - f'_{\x_2}(\ela_1,
    \ela_2))}_{\to 0 \text{ по непрерывности}}\frac{\x_2 - \ela_2}{\norm{\x -
    \ela}}\right]\norm{\x - \ela} \\
    &= f'_{\x_1}(\ela_1, \ela_2)(\x_1 - \ela_1) + f'_{\x_2}(\ela_1, \ela_2)(\x_2
    - \ela_2) + \a(\x - \ela)\norm{\x - \ela}
\end{align*}
\end{proof}
