\chapter{Функциональные последовательности и ряды}

\section{Поточечная и равномерная сходимости последовательностей функций}

\begin{remark}
    \textit{Здесь и далее запись вида $f \to \bot$ будет означать, что $f$
    сходится. Знак $\bot$ используется, если не важно (или не известно), к чему
    сходится $f$.}
\end{remark}

\begin{definition}
    $f_n \colon E \to \R$ \textit{сходится поточечно} к $f \colon E \to \R$ на
    $E$, если
\[
    \forall x_0 \in E~~ f_n(x_0) \to f(x_0)
\]
    иными словами, раскрывая определение сходимости последовательности:
\[
    \forall x_0 \in E~ \left[ \forall \e > 0~ \exists N \in \mathbb{N}\colon~
    \forall n > N~~ |f_n(x_0) - f(x_0)| < \e \right]
\]
    Обозначение: $f_n \to f$.
\end{definition}

\begin{examples}
    TBD
\end{examples}

\begin{definition}
    $f_n \colon E \to \R$ \textit{сходится равномерно} к $f \colon E \to \R$ на
    $E$, если
\[
    \sup_{\x \in E}|f_n(x) - f(x)| \xrightarrow[n \to +\infty]{} 0
\]
    или, раскрывая описание супремума
\[
    \forall \e > 0~ \exists N \in \mathbb{N}\colon~ \forall n > N~ [\forall x \in
    E~~ |f_n(x) - f(x)| < \e]
\]
    Обозначение: $f_n \rcon f$.
\end{definition}

\begin{remark}
    Из равномерной сходимости очевидным образом следует поточечная:
\[
    f_n \rcon f \Lra f_n \to f
\]
\end{remark}

\textit{Про сходимость мы значем очень многое для случая метрических пространств.
А нельзя ли переформулировать новые определения так, чтобы они оказались обычной
сходимостью, просто в хитром метрическом пространстве?}
\begin{present}(Метрическое пространство ограниченных функций)

    Положим
\[
    \mathcal{F} \defeq \{\, X \to \mathbb{R} \mid f \text{ограничено} \,\}
\]
    На этом множестве тривиально задается структура линейного пространства:
\begin{gather*}
    (f + g)(x) = f(x) + g(x) \\
    (\lambda f)(x) = \lambda f(x)
\end{gather*}

    Оказывается, можно ввести \textbf{метрику} на $\mathcal{F}$, сходимость
    по которой есть равномерная сходимость. Для $f\!, \! g \in \mathcal{F}$
    положим
\[
    \r(f, g) \defeq \sup_{x \in X}|f(x) - g(x)|
\]
    Проверим, что это --- метрика на $\mathcal{F}$

    \begin{itemize}
        \item[i)] Неотрицательность очевидна. Равенство нулю может выполнится
        только для равных функций.
        \item[ii)] Симметричность очевидна.
        \item[iii)] Проверим неравенство треугольника. Применим техническое
        описание супремума для $\r(f_1, f_2)$:
\[
            \forall \e > 0~ \exists x \colon~ \sup_{y \in X}|f_1(y) - f_2(y)| - \e
            \leqslant |f_1(x) - f_2(x)|
\]
        Далее
        \begin{align*}
            \forall \e > 0~ \exists x \colon~  \sup_{y \in X}|f_1(y) - f_2(y)| -
            \e &\leqslant |f_1(x) - f_2(x)| \leqslant |f_1(x) - f_3(x)| + |f_3(x)
            - f_2(x)| \\ &\leqslant \sup_{y \in X}|f_1(y) - f_3(y)| + \sup_{y \in
            X}|f_2(y) - f_3(y)| \\ &= \r(f_1, f_3) + \r(f_2, f_3)
        \end{align*}
        Получаем
\[
    \forall \e > 0~~ \r(f_1, f_2) - \e \leqslant \r(f_1, f_3)
    + \r(f_2, f_3)
\]
        Откуда непосредственно следует
\[
    \r(f_1, f_2) \leqslant \r(f_1, f_3) + \r(f_2, f_3)
\]
    \end{itemize}

    Осталось только понять, что теперь означает сходимость по этой метрике.
    Пусть $(f_n)$ --- последовательность в $\mathcal{F}$, сходящаяся к $f$ по
    метрике $\r$:
\[
    \forall \e > 0~ \exists N \in \mathbb{N}\colon~ \forall n > N~~ \r(f_n, f) <
    \e
\]
    Раскроем значение $\r$:
\[
    \forall \e > 0~ \exists N \in \mathbb{N}\colon~ \forall n > N~~ [\forall x
    \in X~~ |f_n(x) - f(x)| < \e]
\]
    А это --- обычное определение равномерной сходимости!
\end{present}

\textit{Подобную конструкцию, по всей видимости, не получится ввести для
поточечной сходимости. Зато, можно построить хаусдорфово топологическое
пространство, в котором сходимость будет означать поточечную сходимость.}
\begin{present}(Топологическое пространство ограниченных функций)

    Введем на $\mathcal{F}$ топологию, порожденную следующими множествами:
\[
    U_{\e}(f)_{x_1, \ldots, x_n} \defeq
    \{\, g \colon X \to \R \mid \forall i~~ |g(x_i) - f(x_i)| < \e \,\}
\]
    Поймем теперь, что означает сходимость в этом топологическом пространстве:
\[
        f_n \to f \Llra \forall U_{\e}(f)_{x_1, \ldots, x_n} \exists N \in
        \mathbb{N}\colon~ \forall n > N~~ f_n \in U_{\e}(f)_{x_1, \ldots, x_n}
\]
    Что означает
\[
    \forall \e > 0~ \exists N \in \mathbb{N}\colon~ \forall n > N~~ \forall i~~
    |f_n(x_i) - f(x_i)| < \e
\]
    Что как раз и есть поточечная сходимость! Просто запись вида
\[
    [\forall x_0 \in X~ \forall \e > 0]~ \exists N \in \mathbb{N}\colon~ \forall
    n > N~~ |f_n(x_0) - f(x_0)| < \e
\]
    В этом пространстве обретает вид
\[
    [\forall U_{\e}(f)_{x_0}]~ \exists N \in \mathbb{N}\colon~ \forall
    n > N~~ |f_n(x_0) - f(x_0)| < \e
\]
\end{present}

\begin{theorem}(Критерий Больцано-Коши равномерной сходимости)

\[
    f_n \rcon f \Llra \forall \e > 0~ \exists N \in \mathbb{N}\colon~
    \forall n, m > N~ [\forall x~ |f_n(x) - f_m(x)| < \e]
\]
\end{theorem}
\begin{proof}
    \enewline
    \begin{itemize}
        \item[$\Lra$] Обычное свойство всех последовательностей, сходящихся по
        метрике (если все $f_n$ и $f$ лежат в $\mathcal{F}$). Общее
        доказательство такое:
\[
            |f_n(x) - f_m(x)| \leqslant |f_n(x) - f(x)| + |f(x) - f_m(x)| <
            \frac{\e}{2} + \frac{\e}{2} = \e
\]

        \item[$\Lla$] Зафиксируем $x$. Тогда $f_n(x)$ --- обычная фундаментальная
        вещественная последовательность. Тогда, так как $\mathbb{R}$ --- полное,
        получаем
\[
        \forall x~ \exists \lim_{n \to +\infty}f_n(x) =: f(x)
\]
        Покажем, что $f_n \rcon f$. Посмотрим на фундаментальность $f_n$:
\[
        \forall \e > 0~ \exists N \in \mathbb{N}\colon~ \forall n, m > N~~
        [\forall x~ |f_n(x) - f_m(x)| < \e]
\]
        и перейдем к пределу $m \to +\infty$:
\[
        \forall \e > 0~ \exists N \in \mathbb{N}\colon~ \forall n > N~~
        [\forall x~ |f_n(x) - f(x)| < \e]
\]
        Что и есть определение равномерной сходимости.
    \end{itemize}
\end{proof}

\begin{examples}
    TBD
\end{examples}

\section{Теоремы о равномерно сходящихся функциональных последовательностях}

\begin{theorem}(Стокс-Зейдель)

    Пусть $f_n, f \colon X \to \R$, $X$ --- топологическое пространство, $f_n$
    непрерывны в $c \in E$, и $f_n \rcon f$ на $X$. Тогда $f$ непрерывна в $c$.
    Иначе говоря, коммутативна следующая диаграмма:

    \xymatrix{& & & & & & f_n(x) \ar@<-.5ex>[r] \ar@<.5ex>[r]
    \ar[d]^{x \to c} & f(x) \ar@{-->}[d]^{x \to c} \\
              & & & & & & f_n(c) \ar[r]^{n \to +\infty} & f(c)}
\end{theorem}
\begin{proof} Для любых $n$ выполнено
\[
    |f(x) - f(y)| \leqslant |f(x) - f_n(x)| + |f_n(x) - f_n(y)| + |f_n(y) - f(y)|
\]
    Воспользуемся равномерной сходимостью: выберем $n$ таким, чтобы
\begin{align*}
    |f(x) - f_n(x)| < \e \\
    |f_n(y) - f(y)| < \e
\end{align*}
    Теперь воспользуемся непрерывностью $f_n$: выберем такую окрестность
    $U(c)$, чтобы $\forall x, y \in U(c)$
\[
    |f_n(x) - f_n(y)| < \e
\]
     Тогда
\[
    |f(x) - f(y)| < 3\e
\]
    что и означает непрерывность $f$ в точке $c$.
\end{proof}

\begin{definition}
    Будем говорить, что $f_n$ сходится \textit{локально равномерно} к $f$ на
    $X$, если
\[
    \forall x \in X~ \exists U(x)\colon~ f_n \rcon f \text{ на } U(x)
\]
\end{definition}

\begin{remark}
    Для выполнения условия теоремы Стокса-Зейделя достаточно равномерной
    сходимости на некоторой окрестности $c$.
\end{remark}

\begin{remark}
    Для того, чтобы $f$ было непрерывным на $X$, достаточно, чтобы
    $f_n$ локально равномерно на $X$ сходилось к $f$.
\end{remark}

\begin{theorem}(О предельном переходе под знаком интеграла)

    Пусть $f_n \in C([a, b])$, $f_n \rcon f$ на $[a, b]$. Тогда
\[
    \int_a^b{f_n} \xrightarrow[n \to +\infty]{} \int_a^b{f}
\]
    иначе говоря, коммутативна следующая схема:

    \xymatrix{& & & & & & f_n \ar@<-.5ex>[r] \ar@<.5ex>[r]
    \ar[d]^{\int{}} & f \ar[d]^{\int{}} \\
              & & & & & & \int_a^b{f_n} \ar@{-->}[r]^{n \to +\infty} &
              \int_a^b{f}}
\end{theorem}
\begin{proof}
    $f$ непрерывна на $[a, b]$ по теореме Стокса-Зейделя, поэтому интеграл имеет
    смысл. Тогда
\[
    \left|\int_a^b{f_n} - \int_a^b{f}\right| \leqslant \int_a^b{|f_n - f|}
    \leqslant \max_{x \in [a, b]}{|f_n(x) - f(x)|} \cdot |b - a|
\]
    Из равномерной сходимости имеем:
\[
    \max_{x \in [a, b]}{|f_n(x) - f(x)|} \xrightarrow[n \to +\infty]{} 0
\]
    тогда
\[
    \left|\int_a^b{f_n} - \int_a^b{f}\right| \leqslant \max_{x \in [a,
    b]}{|f_n(x) - f(x)|} \cdot |b - a| \xrightarrow[n \to +\infty]{} 0
\]
\end{proof}

\begin{theorem}(Правило Лейбница)

    $f \colon [x_1, x_2] \times [y_1, y_2] \to \R$, $\exists f'_y$, $f'_y, f$
    непрерывны. Пусть
\[
    \phi(y) = \int_{x_1}^{x_2}{f(x, y) dx}
\]
    Тогда $\phi$ дифференцируемо на $[y_1, y_2]$ и
\[
    \phi'(y) = \int_{x_1}^{x_2}{f'_y(x, y) dx}
\]
\end{theorem}
\begin{proof}
\[
    \frac{\phi(y + \frac{1}{n}) - \phi(y)}{\frac{1}{n}} =
    \frac{1}{n} \int_{x_1}^{x_2}{\left(f\left(x, y + \frac{1}{n}\right) - f(x,
    y)\right) dx}
    \underset{\text{Лагранж}}{=} \int_{x_1}^{x_2}{f'_y\left(x, y +
    \frac{\theta}{n}\right) dx}
\]
    Обозначим
\[
    g_n(x, y) = \int_{x_1}^{x_2}{f'_y\left(x, y + \frac{\theta}{n}\right) dx}
\]
    $f'_y$ непрерывно на компакте, поэтому равномерно непрерывна на нём.
    Воспользуемся этим:
\[
    \forall \e~ \exists \delta~ \forall n\colon \frac{1}{n} < \delta~
    \forall x~~ \left| f'_y\left(x, y + \frac{1}{n}\right) - f'_y(x, y) \right|
    < \e
\]
    Отсюда получаем по определению
\[
    g_n(x, y) \rcon f'_y(x, y) \text{ на } [x_1, x_2]
\]
    Воспользуемся теоремой о предельном переходе под знаком интеграла:
\[
    \int_{x_1}^{x_2}{g_n(x, y)} \xrightarrow[n \to +\infty]{}
    \int_{x_1}^{x_2}{f'_y(x, y) dx}
\]
    Понятно, что вместо последовательности $\frac{1}{n}$ можно рассматривать
    любую последовательность $h_n$, сходящуюся к $0$. То есть
\[
    \phi'(y) = \lim_{n \to +\infty}{\frac{\phi(y + h_n) - \phi(y)}{h_n}} =
    \int_{x_1}^{x_2}{f'_y(x, y) dx}
\]
\end{proof}

\begin{theorem}(О предельном переходе под знаком производной)

    $f_n \in C^1([a, b])$, $f_n \to f$ поточечно на $\langle a, b \rangle$,
    $f'_n \rcon \f$ на $\langle a, b \rangle$. Тогда
    \begin{itemize}
        \item $f \in C^1([a, b])$
        \item $f' = \f$
    \end{itemize}
    иначе говоря, коммутативна следующая схема:

    \xymatrix{& & & & & & f_n \ar[r]^{n \to +\infty}
    \ar[d]^{\der{}{}} & f \ar@{-->}[d]^{\der{}{}} \\
              & & & & & & f'_n \ar@<-.5ex>[r] \ar@<.5ex>[r] & \f}
\end{theorem}
\begin{proof}
    Пусть $x_0, x_1 \in \langle a, b \rangle$, тогда $f'_n \rcon \f$ на $[x_0,
    x_1]$. Тогда по теореме о предельном переходе под знаком интеграла:
\[
    \int_{x_0}^{x_1}{f'_n} \xrightarrow[n \to +\infty]{} \int_{x_0}^{x_1}{\f}
\]
    Откуда
\[
    f(x_1) - f(x_0) \xleftarrow[n \to +\infty]{} f_n(x_1) - f_n(x_0)
    \xrightarrow[n \to +\infty]{} \int_{x_0}^{x_1}{\f}
\]
    То есть
\[
    \int_{x_0}^{x_1}{\f} = f(x_1) - f(x_0)
\]
    Тогда $f$ --- первообразная $\f$. $\f$ непрерывна по теореме Стокса-Зейделя.
    Тогда получаем, что $f \in C^1([a, b])$ и $f' = \f$.
\end{proof}
